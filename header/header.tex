% Genereller Header
\documentclass[10pt,twoside,a4paper,fleqn]{article}
% Dateiencoding
\usepackage[utf8]{inputenc}
% Seitenränder
\usepackage[left=5mm,right=5mm,top=5mm,bottom=5mm,includeheadfoot]{geometry}
% Sprachpaket
\usepackage[ngerman]{babel,varioref}

% Pakete
\usepackage{
	amssymb,
	amsmath,
	fancybox,
	graphicx,
	lastpage,
	wrapfig,
	fancyhdr,
	hyperref,
	verbatim,
	floatflt,
	multicol,
	multirow,
	rotating,
	tabularx,
	pdflscape,
	array,
	longtable,
	listings}

\usepackage{tikz}

% Zum Bilder einfach in Tabellen einfügen (valign=t)
\usepackage[export]{adjustbox}

%%%%%%%%%%%%%%%%%%%%
% Generelle Makros %
%%%%%%%%%%%%%%%%%%%%
\newcommand{\skript}[1]{$_{\textcolor{red}{\mbox{\small{Skript S. #1}}}}$}
\newcommand{\verweis}[2]{\small{(siehe auch \ref{#1}, #2 (S. \pageref{#1}))}}
\newcommand{\verweiskurz}[1]{(\small{siehe \ref{#1}\normalsize)}}
\newcommand{\subsubadd}[1]{\textcolor{black}{\mbox{#1}}}
\newcommand{\formelbuch}[1]{$_{\textcolor{red}{\mbox{\small{S#1}}}}$}

\newcommand{\kuchling}[1]{$_{\textcolor{red}{\mbox{\small{Kuchling #1}}}}$}
\newcommand{\stoecker}[1]{$_{\textcolor{orange}{\mbox{\small{Stöcker #1}}}}$}
\newcommand{\sachs}[1]{$_{\textcolor{blue}{\mbox{\small{Sachs S. #1}}}}$}
\newcommand{\hartl}[1]{$_{\textcolor{green}{\mbox{\small{Hartl S. #1}}}}$}

\newcommand{\schaum}[1]{\tiny Schaum S. #1}

\newcommand{\skriptsection}[2]{\section{#1 {\tiny Skript S. #2}}}
\newcommand{\skriptsubsection}[2]{\subsection{#1 {\tiny Skript S. #2}}}
\newcommand{\skriptsubsubsection}[2]{\subsubsection{#1 {\tiny Skript S. #2}}}

\newcommand{\matlab}[1]{\footnotesize{(Matlab: \texttt{#1})}\normalsize{}}

%eigene Befehle---------------------------------------------------------
\newcommand{\bronstein}[1]{$_{\textcolor{violet}{\mbox{\small{Bronstein S. #1}}}}$}


%%%%%%%%%%
% Farben %
%%%%%%%%%%
\usepackage{xcolor}

%%%%%%%%%%%%%%%%%%%%%%%%%%%%
% Mathematische Operatoren %
%%%%%%%%%%%%%%%%%%%%%%%%%%%%
\DeclareMathOperator{\sinc}{sinc}
\DeclareMathOperator{\sgn}{sgn}
\DeclareMathOperator{\Real}{Re}
\DeclareMathOperator{\Imag}{Im}
%\DeclareMathOperator{\e}{e}
\DeclareMathOperator{\cov}{cov}
\DeclareMathOperator{\PolyGrad}{PolyGrad}

%Makro für 'd' von Integral- und Differentialgleichungen 
\newcommand*{\diff}{\mathop{}\!\mathrm{d}}


%%%%%%%%%%%%%%%%%%%%%%%%%%%
% Fouriertransformationen %
%%%%%%%%%%%%%%%%%%%%%%%%%%%
\usepackage{trfsigns, trsym}
%\unitlength1cm
% Zeitbereich -- Frequenzbereich
%\newcommand{\laplace}
%{
%\begin{picture}(1,0.5)
%\put(0.2,0.1){\circle{0.14}}\put(0.27,0.1){\line(1,0){0.5}}\put(0.77,0.1){\circle*{0.14}}
%\end{picture}
%}
% Frequenzbereich -- Zeitbereich
%\newcommand{\Laplace}
%{
%\begin{picture}(1,0.5)
%\put(0.2,0.1){\circle*{0.14}}\put(0.27,0.1){\line(1,0){0.45}}\put(0.77,0.1){\circle{0.14}}
%\end{picture}
%}


% Fouriertransformationen
\unitlength1cm
\newcommand{\FT}
{
\begin{picture}(1,0.5)
\put(0.2,0.1){\circle{0.14}}\put(0.27,0.1){\line(1,0){0.5}}\put(0.77,0.1){\circle*{0.14}}
\end{picture}
}


\newcommand{\IFT}
{
\begin{picture}(1,0.5)
\put(0.2,0.1){\circle*{0.14}}\put(0.27,0.1){\line(1,0){0.45}}\put(0.77,0.1){\circle{0.14}}
\end{picture}
}




%%%%%%%%%%%%%%%%%%%%%%%%%%%%
% Allgemeine Einstellungen %
%%%%%%%%%%%%%%%%%%%%%%%%%%%%
%PDF Info
\hypersetup{pdfauthor={\authorinfo},pdftitle={\titleinfo},colorlinks=false}
\author{\authorinfo}
\title{\titleinfo}


%%%%%%%%%%%%%%%%%%%%%%%
% Kopf- und Fusszeile %
%%%%%%%%%%%%%%%%%%%%%%%
\pagestyle{fancy}
\fancyhf{}
%Linien oben und unten
\renewcommand{\headrulewidth}{0.5pt} 
\renewcommand{\footrulewidth}{0.5pt}

\fancyhead[L]{\titleinfo{ }\tiny{(\versioninfo)}}
%Kopfzeile rechts bzw. aussen
\fancyhead[R]{Seite \thepage { }von \pageref{LastPage}}
%Fusszeile links bzw. innen
\fancyfoot[L]{\footnotesize{\authorinfo}}
%Fusszeile rechts bzw. ausen
\fancyfoot[R]{\footnotesize{\today}}
%Lizenz CC-BY-NC-SA
% Headerfile für die Einbindung einer Lizenzgrafik in den Footer
% Verwendung: \lizenz{cc-by-nc-sa}{small}
\newcommand{\lizenz}[2]
{
\fancyfoot[C]{
  \includegraphics[width=1.6cm]{./header/lizenzen/#1/#2.png}
}
}
\lizenz{cc-by-nc-sa}{small}
%Einrücken verhindern versuchen
\setlength{\parindent}{0pt}

% Zeilenhöhe Tabellen:
\newcommand{\arraystretchOriginal}{1.5}
\renewcommand{\arraystretch}{\arraystretchOriginal}

