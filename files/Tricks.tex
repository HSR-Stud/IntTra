\section{Riisä Tricks und Merksätze}
\begin{itemize}
  \item $H(s)$ = UTF = Laplacetransformierte der Impulsantwort ($y_\delta(t)$)
  \item $H(j \omega)$ = Frequenzgang = UTF auf imaginärer Achse
  \item $y_\sigma(t) = \int\limits_0^t y_\delta(u)du$
  \item $\left| \frac{ja + b}{a^2 + b^2} \right| = \sqrt{\frac{1}{a^2 + b^2}}$
  \item Dirac-Funktion: $s(t)\delta(t-t_0) = s(t_0)\delta(t-t_0)$
  \item Antwort eines LTI-Systems auf eine harmonisches Schwingung mit Frequenz $\omega \Rightarrow$ harmonische
  Schwingung mit gleicher Frequenz aber anderer Aplitude und Phase ($\mathcal{L}\{e^{j \omega t}\} = H(\omega) \cdot
  e^{j \omega t}$)
  \item $H(\omega)$ = komplexwertige Funktion der Frequenz $\omega$, die für jede Frequenz $\omega$ die
  Änderung von Amplitude und Phase durch das System speichert = Frequenzgang = Antwort auf harmonische Schwingung
  beliebiger Frequenz
\end{itemize}