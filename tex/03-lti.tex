\section{Lineare Zeitinvariante Systeme}
\begin{center}
  \begin{tikzpicture}[
      system/.style = {draw, thick, inner sep = 4mm, outer sep = 1mm}
    ]
    \matrix[row sep=3mm, column sep=1.5cm] (M) {
      \node (x) {\(x(t)\)}; &
      \node (g) {\(g(t) = y_\delta (t)\)}; &
      \node (y) {\(y(t) = g(t) * x(t)\)}; \\

      &
      \node (h) {\(h(t)\)}; &
      \node (yw) {\(y_\omega(t) = h(t) * x(t)\)}; \\

      \node (in) {Anregung}; &
      \node[system, fill=white] (sys) {LTI-System \(\mathcal{S}\)}; &
      \node (out) {Antwort}; \\

      \node (X) {\(X(s)\)}; &
      \node (G) {\(G(s) = 1/p(s)\)}; &
      \node (Y) {\(Y(s) = G(s) \cdot X(s)\)}; \\

      \node (Xw) {\(X(\omega)\)}; &
      \node (H)  {\(H(\omega) = G(j\omega)\)}; &
      \node (Yw) {\(Y_\omega (\omega) = H(\omega) \cdot X(\omega)\)}; \\
    };

    \draw[thick, ->] (in) to (sys);
    \draw[thick, ->] (sys) to (out);

    \begin{pgfonlayer}{background}
      \coordinate (T1) at ($(x.north west) - (.8,-.1)$);
      \coordinate (T2) at ($(yw.south east) + (.8,-.1)$);

      \coordinate (B1) at ($(X.north west) - (0,-.1)$);
      \coordinate (B2) at ($(Y.south east) + (0,-.1)$);

      \coordinate (F1) at ($(Xw.north west) - (0,-.1)$);
      \coordinate (F2) at ($(Yw.south east) + (0,-.1)$);

      \fill[color=blue!20] (T1) rectangle (T2);
      \fill[color=magenta!20] (B1 -| T1) rectangle (B2 -| T2);
      \fill[color=red!20] (F1 -| T1) rectangle (F2 -| T2);
      % \fill[top color=blue!20, bottom color=magenta!20]
        % (T1) rectangle (B2);
    \end{pgfonlayer}
  \end{tikzpicture}
\end{center}
Ein System mit der folgenden Eigenschaften ist als LTI bezeichnet.
\begin{center}
  \begin{tabularx}{\linewidth}{l >{\(\displaystyle }X<{\)}}
    Linear & \mathcal{S} (ax_1 + bx_2) = a\mathcal{S} x_1 + b\mathcal{S} x_2 = ay_1 + by_2 \\
    Zeitinvariant & \mathcal{S} x(t + t_0) = y(t + t_0) \\
  \end{tabularx}
\end{center}
LTI Systeme sind matematisch mit linearer DGL mit konstanten Koeffizienten beschreibt.
\[
  x = \sum_{k=0}^n a_k y^{(k)} = a_n y^{(n)} + a_{n-1} y^{(n-1)}+ \cdots + a_0 y
\]

\subsection{Impulsantwort}
\subsection{Frequenzgang}
\subsection{Eigenschwingungen}
\subsection{Station\"are L\"osung}
