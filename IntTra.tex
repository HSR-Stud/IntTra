% !TeX program = xelatex
% !TeX encoding = utf8
% !TeX root = IntTra.tex

%% TODO: publish to CTAN
\documentclass[margin=normal]{tex/hsrzf}

%%%%%%%%%%%%%%%%%%%%%%%%%%%%%%%%%%%%%%%%%%%%%%%%%%%
% Packages

%% TODO: publish to CTAN
\usepackage{tex/hsrstud}
\usepackage{tex/inttra}

%% Language configuration
\usepackage{polyglossia}
\setdefaultlanguage[variant=swiss]{german}

%% License configuration
\usepackage[
    type={CC},
    modifier={by-nc-sa},
    version={4.0},
    lang={german},
]{doclicense}

%% Place figures
\usepackage{float}

%% Nicer tables
\usepackage{tabularx}
\usepackage{array}
\usepackage{booktabs}

%% Pretty drawings and equations
\usepackage{tikz}
\usetikzlibrary{backgrounds}
\usetikzlibrary{calc}

%% Landscape contents
\usepackage{rotating}

%%%%%%%%%%%%%%%%%%%%%%%%%%%%%%%%%%%%%%%%%%%%%%%%%%%
% Metadata

\course{Elektrotechnik}
\module{IntTra}
\semester{Herbstsemester 2020}

\authoremail{naoki.pross@ost.ch}
\author{\textsl{Naoki Pross} -- \texttt{\theauthoremail}}

\title{\texttt{\themodule} Zusammenfassung}
\date{\thesemester}

%%%%%%%%%%%%%%%%%%%%%%%%%%%%%%%%%%%%%%%%%%%%%%%%%%%
% Document

\begin{document}

\maketitle
\tableofcontents
\section*{Lizenz}
\doclicenseThis

\clearpage

\section{Zusammenhang zwischen Integraltransformationen}
\begin{figure}[H] \centering
  \begin{tikzpicture}[semithick]

  \node[fill=blue!20,
        minimum height=5cm,
        minimum width=5cm] (time) at (0,0) {};

  \node[fill=red!20,
        minimum height=5cm,
        minimum width=4cm,
        anchor=east] (freq) at (-3,0) {};

  \node[fill=magenta!20,
        minimum height=5cm,
        minimum width=4cm,
        anchor=west] (cfreq) at (3,0) {};

  \node[blue!80!black, above] at (time.north) {Zeitbereich};
  \node[red!80!black, above] at (freq.north) {Frequenzbereich};
  \node[magenta!60!black, above] at (cfreq.north) {Bildbereich};

  \node[anchor=south west, rotate=-90, magenta!80!black]
    at (cfreq.north east) {\(z\)-Ebene};
  \node[anchor=south east, rotate=-90, magenta!80!black]
    at (cfreq.south east) {\(s\)-Ebene};

  \node (xt) at ($(time)  - (0,1.75)$) {\(f(t)\)};
  \node (xn) at ($(time)  + (0,1.75)$) {\(y_k\)};

  \node (Xo) at ($(freq)  + (-.75,-1.75)$) {\(F(\omega)\)};
  \node (Xn) at ($(freq)  + (-.75,1.75)$) {\(\hat{c}_k\)};
  \node (cn) at ($(freq)  + (-.75,0)$)    {\(c_k\)};

  \node (Xs) at ($(cfreq) - (0,1.75)$) {\(F(s)\)};
  \node (Xz) at ($(cfreq) + (0,1.75)$) {\(F(z)\)};

  \draw[->] (xt) to node[midway, right] {\(\mathcal{S}_\delta\)} (xn);

  \draw[<->] (Xs) to node[midway, right] {\(z=e^{sT}\)} (Xz);

  \draw[<->] (cn) to
    node[near end, left, anchor=east] {\(\displaystyle\lim_{T \to \infty}\)}
    node[near start, right] {\(F(k\omega_0) = 2Tc_k\)} (Xo);

  \draw[->] (Xn) to
    node[midway, left, anchor=east] {\(\displaystyle\lim_{N \to \infty}\)} (cn);

  \draw[<->] (xt) to[out=135, in=10]
    node[near end, sloped, above] {\(\mathcal{F}_T\)}
    node[near start, sloped, above] {\(\mathcal{F}_T^{-1}\)} (cn);

  \draw[<->] (xt) to
    node[near end, above] {\(\mathcal{F}\)}
    node[near start, above] {\(\mathcal{F}^{-1}\)} (Xo);

  \draw[<->] (xt) to
    node[near end, above] {\(\mathcal{L}\)}
    node[near start, above] {\(\mathcal{L}^{-1}\)} (Xs);

  \draw[<->] (xn) to
    node[near end, above] {\(\mathtt{DFT}_N\)}
    node[near end, below] {\(\mathtt{FFT}_N\)}
    node[near start, above] {\(\mathtt{IDFT}_N\)}
    node[near start, below] {\(\mathtt{IFFT}_N\)} (Xn);

  \draw[<->] (xn) to
    node[near end, above] {\(\mathcal{Z}\)}
    node[near start, above] {\(\mathcal{Z}^{-1}\)} (Xz);

  \draw[->] (Xs) .. controls ($(Xs)-(0,2)$) and ($(Xo)-(0,2)$) ..
    node[midway, above] {\(s = j\omega\)} (Xo);
\end{tikzpicture}

\end{figure}

\section{Fourier}
\subsection{Reihe, Transformation und Diskrete Transformation}
\[
  \everymath={\displaystyle}
  \begin{array}{l l @{\hspace{1cm}} l l}
    c_k = \fourier_T f &=
    \frac{1}{T} \int\limits_{-T/2}^{T/2} f(t) \, e^{-jk\omega_0 t} \di{t}
    &
      f(t) = \ifourier_T \left( c_k \right)_{k\in\mathbf{Z}} &=
      \sum_{k=-\infty}^{\infty} c_k e^{jk\omega_0 t}
      \\

    F(\omega) = \fourier f &=
    \int\limits_{-\infty}^{\infty} f(t) \, e^{-j\omega t} \di{t}
    &
      f(t) = \ifourier F &=
      \frac{1}{2\pi} \int\limits_{-\infty}^\infty F(t) e^{j\omega t} \di{\omega}
      \\

    \hat{c}_k = \mathtt{DFT}_N \left( y_n \right) &=
    \sum_{k=0}^{N-1} \hat{c}_k e^{-jkn2\pi /N}
      &
      y_n = \mathtt{IDFT}_N \left( \hat{c}_k \right) &=
      \frac{1}{N}\sum_{n=0}^{N-1} y_n e^{jkn2\pi /N}
  \end{array}
\]

\subsection{Hilbertstransformation}
\[
  \hilbert x(t) = \frac{1}{\pi} \int\limits_{-\infty}^{\infty}
    \frac{x(u)}{t - u} \di{u}
  \qquad
  \hilbert \left\{ X(\omega) \right\}
    = \hilbert \left\{ \Re(\omega) + j\Im(\omega) \right\}
    = - \Im(\omega) + j\Re(\omega)
\]

\section{Laplace Transformation}
\subsection{Z-Transformation}

\section{Lineare Differenzialgleichungen}
\begin{figure}[H] \centering
  \begin{tikzpicture}[
    semithick,
    mtext/.style = {inner sep=3mm},
  ]

  \matrix[row sep=1.25cm, column sep=.75cm] (M) {
    \node[mtext] (d-image-dgl) {\(pY - h = F\)}; &&
      \node[mtext] (time-dgl) {\(\sum_k a_k y^{(k)} = f\)}; &&
      \node[mtext] (i-image-dgl) {\(pY - h = F\)}; &
      \node (A) {}; \\

    &&
      &&
      \node[mtext] (i-image-dgl-g) {\(Y_G = G\cdot F\)}; &
      \node[mtext] (i-image-dgl-h) {\(Y_H = h/p\)}; \\

    \node[mtext] (d-image-sol) {\(Y = G(F + h)\)}; &&
      \node[mtext] (time-sol) {\(y\)}; &&
      \node[mtext] (i-image-sol) {\(Y = Y_G + Y_H\)}; &
      \node (B) {}; \\
  };

  % time domain solution
  \draw[->, dashed] (time-dgl) to (time-sol);

  % direct laplace solution
  \draw[->] (time-dgl) to node[midway, above] {\(\mathcal{L}\)} (d-image-dgl);

  \draw[->] (d-image-dgl) to node[midway, right] {\(G = 1/p\)} (d-image-sol);

  \draw[->] (d-image-sol) to node[pos=.3, above] {\(\mathcal{L}^{-1}\)} (time-sol);

  % indirect laplace solution
  \draw[->] (time-dgl) to node[midway, above] {\(\mathcal{L}\)} (i-image-dgl);

  \draw[->] (i-image-dgl) .. controls (A) ..
    node[near end, right] (F) {\(F = 0\)} (i-image-dgl-h);

  \draw[->] (i-image-dgl) to
    node[midway, right] {\(h = 0\)} (i-image-dgl-g);

  \draw[->] (i-image-dgl-g) to (i-image-sol);
  \draw[->] (i-image-dgl-h) .. controls (B) .. (i-image-sol);

  \draw[->] (i-image-sol) to node[pos=.25, above] {\(\mathcal{L}^{-1}\)} (time-sol);

  \begin{pgfonlayer}{background}
    \coordinate (T1) at (time-dgl.north west);
    \coordinate (T2) at (time-dgl.east |- time-sol.south east);

    \coordinate (iI1) at (i-image-dgl.north west);
    \coordinate (iI2) at (F.east |- B.south east);

    \coordinate (dI1) at (d-image-dgl.north west);
    \coordinate (dI2) at (d-image-sol.east);

    % add a bit more space
    \coordinate (T1) at ($(T1) + (-.3,.3)$);
    \coordinate (T2) at ($(T2) + (.3,-.3)$);

    \coordinate (iI1) at ($(iI1) + (-.3,.3)$);
    \coordinate (iI2) at ($(iI2) + (.3,-.3)$);

    \coordinate (dI1) at ($(dI1) + (-.3,.3)$);
    \coordinate (dI2) at ($(dI2) + (.3,-.3)$);

    % adjust heights of I to match T
    \coordinate (iI1) at (T1 -| iI1);
    \coordinate (iI2) at (T2 -| iI2);

    \coordinate (dI1) at (T1 -| dI1);
    \coordinate (dI2) at (T2 -| dI2);

    \node[above right, color=blue!70!black] at (T1) {Zeitbereich};
    \node[above right, color=magenta!70!black] at (iI1) {Bildbereich};
    \node[above right, color=magenta!70!black] at (dI1) {Bildbereich};

    % \node[below left] at (dI2) {Direkte L\"osung};
    % \node[below left] at (iI2) {Indirekte L\"osung};

    \fill[color=blue!20] (T1) rectangle (T2);
    \fill[color=magenta!20] (iI1)rectangle (iI2);
    \fill[color=magenta!20] (dI1) rectangle (dI2);
  \end{pgfonlayer}
\end{tikzpicture}

  \caption{
    Schematische Darstellung des Prozesses um lineare Differenzialgleichungen
    zu l\"osen.
  }
\end{figure}

\subsection{Laplace Transformierte der Ableitung}
\[
  \laplace\left\{ y^{(k)} \right\} =
  s^k Y(s) - \sum_{j = 0}^{k - 1} s^{k-j} y^{(j)}(0)
\]

\subsection{Partialbruchzerlegung}
Seien \(P\) und \(Q\) Polynomen von \(s\).
Der Nenner \(Q(s)\) kann in einem Produkt von \(n\) linearen und \(m\)
quadratischen Terme faktorisiert werden, dass heisst mit Terme von der Form
\(\ell (s) = s - r\) und \(q(s) = a s^2 + b s + c\).
%
Die Partialbruchzerlegung macht im einfachsten Fall (\(Q = \ell\cdot q\)) aus
\(P/(\ell \cdot q)\) eine Summe \(A/\ell + (Bs + C)/q\). Mit mehre Faktoren es
muss zu jedem \(\ell\) und \(q\) einen \emph{eindeutigen} Z\"ahler (\(A\) bzw.
\(Bs + C\)) zugeordnet werden.
%
\begin{align*}
  \frac{P(s)}{Q(s)}
  =
  \frac{P(s)}{
    \colorbox{teal!20}{\(
      \ell_1(s)\cdots \ell_n(s)
    \)}
    \cdot
    \colorbox{purple!20}{\(
      q_1(s)\cdots q_m(s)
    \)}
  }
  &=
  \colorbox{teal!20}{\(\displaystyle
    \sum_{j=1}^n \sum_{k=1}^{u_j} \frac{A_{jk}}{(s - r_j)^k}
  \)}
  +
  \colorbox{purple!20}{\(\displaystyle
    \sum_{j=1}^m \sum_{k=1}^{w_j} \frac{B_{jk} s + C_{jk}}{(a_j s^2 + b_j s + c_j)^k}
  \)}
\end{align*}

Wenn 2 oder allgemeiner \(u\) Faktoren \(\ell_j, \ell_{j+1},\dots,\ell_{j+u}\)
in der Faktorisierung gleich sind, dann findet man sie als \(\ell_j(s)^{u}\).
Auf \"ahnliche Weise wenn \(q_j = \cdots = q_{j+w}\), es gibt dann
\(q_j(s)^{w}\).
%
Diese (\(j\)-te, \((j+1)\)-te, \dots) m\"ussen jedoch verschiedene Z\"ahler haben,
somit potenziert man die (gleiche) Nenner \(\ell_j, \ell_{j+1}, \dots\) mit steigenden Potenzen 
um sie von einander zu unterscheiden.
Sonst, weil
\(\ell_j = \ell_{j+1} \textcolor{gray}{= \cdots = \ell_{j+u}}\), ist
\begin{align*}
  \frac{A_j}{\ell_j} + \frac{A_{j+1}}{\ell_{j+1}} =
    \frac{A_j + A_{j+1}}{\ell_j} = \frac{A'_j}{\ell_j}
  \quad\text{unerw\"unscht! }\implies\quad
  \frac{A_j}{\ell_j} + \frac{A_{j+1}}{\ell_{j+1}^2} =
    \frac{A_{j1}}{\ell_j} + \frac{A_{j2}}{\ell_j^2}
      \begingroup\color{gray}
        + \frac{A_{jk}}{\ell_j^k}
      \endgroup
\end{align*}

Dasselbe ist auch f\"ur \(q_j\) mit \(B_{jk}s + C_{jk}\).
Jedes \(A, B\) und \(C\) wird mittels Koeffizientenvergleich zu \(P(s)\) durch
ein lineares Gleichungssystem (z.B. in Matrixform) bestimmt.
Beispiel: Hier sind
  \colorbox{teal!20}{\(n = 1\)},
  \textcolor{blue}{\(u_1 = 2\)} und
  \colorbox{purple!20}{\(m = 1\)},
  \textcolor{red!80!black}{\(w_1 = 1\)}.
\begin{align*}
  \frac{3}{
    \colorbox{teal!20}{\(
      (s + 3)^{\textcolor{blue}{2}}
    \)}
    \cdot
    \colorbox{purple!20}{\(
      (s^2 + 9)^{\textcolor{red}{1}}
    \)}
  }
  &=
  \colorbox{teal!20}{\(\displaystyle
    \frac{A_{11}}{s + 3} +
    \frac{A_{12}}{(s + 3)^2}
  \)}
  +
  \colorbox{purple!20}{\(\displaystyle
    \frac{B_{11} s + C_{11}}{s^2 + 9}
  \)}
  \\
  3 &= A_{11} (s + 3)(s^2 + 9) + A_{12} (s^2 + 9) + (B_{11} s + C_{11})(s + 3)^2 \\
  \begingroup\color{gray}0s^3 + 0s^2 + 0s^1 +\endgroup 3s^0 &=
    A_{11} (s^3 + 3s^2 + 9s + 27) +
    A_{12} (s^2 + 9) +
    B_{11} (s^3 + 6s^2 + 9s) +
    C_{11} (s^2 +6s + 9)
\end{align*}
Das Gleichungssystem in \(\mathcal{P}_{n+m}(\mathbf{R})\)
\[
  \mathcal{Q}\vec{k} = \vec{p} \iff
  \begingroup\color{lightgray}
    \begin{matrix}
      s^3 \\ s^2 \\ s^1 \\ s^0
    \end{matrix}
  \endgroup
  \begin{bmatrix}
     1 & 0 & 1 & 0 \\
     3 & 1 & 6 & 1 \\
     9 & 0 & 9 & 6 \\
    27 & 9 & 0 & 9 
  \end{bmatrix}
  \begin{bmatrix}
    A_{11} \\ A_{12} \\ B_{11} \\ C_{11}
  \end{bmatrix} =
  \begin{bmatrix}
    0 \\ 0 \\ 0 \\ 3
  \end{bmatrix}
  \quad\implies\quad
  \vec{k} = \mathcal{Q}^{-1} \vec{p} =
    \frac{1}{18}
    \begin{bmatrix}
      1 \\ 3 \\  -1 \\ 0
    \end{bmatrix}
\]
% \(\vec{A}, \vec{B}, \vec{C}, \vec{P} \in \mathcal{P}(\mathbf{C})\)
% \[
%   \begin{bmatrix}
%     \v{A} \\ \v{B} \\ \v{C}
%   \end{bmatrix}
%   = \v{P}
% \]

\section{Lineare Zeitinvariante Systeme}
\begin{center}
  \begin{tikzpicture}[
      system/.style = {draw, thick, inner sep = 4mm, outer sep = 1mm}
    ]
    \matrix[row sep=3mm, column sep=1.5cm] (M) {
      \node (x) {\(x(t)\)}; &
      \node (g) {\(g(t) = y_\delta (t)\)}; &
      \node (y) {\(y(t) = g(t) * x(t)\)}; \\

      &
      \node (h) {\(h(t)\)}; &
      \node (yw) {\(y_\omega(t) = h(t) * x(t)\)}; \\

      \node (in) {Anregung}; &
      \node[system, fill=white] (sys) {LTI-System \(\mathcal{S}\)}; &
      \node (out) {Antwort}; \\

      \node (X) {\(X(s)\)}; &
      \node (G) {\(G(s) = 1/p(s)\)}; &
      \node (Y) {\(Y(s) = G(s) \cdot X(s)\)}; \\

      \node (Xw) {\(X(\omega)\)}; &
      \node (H)  {\(H(\omega) = G(j\omega)\)}; &
      \node (Yw) {\(Y_\omega (\omega) = H(\omega) \cdot X(\omega)\)}; \\
    };

    \draw[thick, ->] (in) to (sys);
    \draw[thick, ->] (sys) to (out);

    \begin{pgfonlayer}{background}
      \coordinate (T1) at ($(x.north west) - (.8,-.1)$);
      \coordinate (T2) at ($(yw.south east) + (.8,-.1)$);

      \coordinate (B1) at ($(X.north west) - (0,-.1)$);
      \coordinate (B2) at ($(Y.south east) + (0,-.1)$);

      \coordinate (F1) at ($(Xw.north west) - (0,-.1)$);
      \coordinate (F2) at ($(Yw.south east) + (0,-.1)$);

      \fill[color=blue!20] (T1) rectangle (T2);
      \fill[color=magenta!20] (B1 -| T1) rectangle (B2 -| T2);
      \fill[color=red!20] (F1 -| T1) rectangle (F2 -| T2);
      % \fill[top color=blue!20, bottom color=magenta!20]
        % (T1) rectangle (B2);
    \end{pgfonlayer}
  \end{tikzpicture}
\end{center}
Ein System mit der folgenden Eigenschaften ist als LTI bezeichnet.
\begin{center}
  \begin{tabularx}{\linewidth}{l >{\(\displaystyle }X<{\)}}
    Linear & \mathcal{S} (ax_1 + bx_2) = a\mathcal{S} x_1 + b\mathcal{S} x_2 = ay_1 + by_2 \\
    Zeitinvariant & \mathcal{S} x(t + t_0) = y(t + t_0) \\
  \end{tabularx}
\end{center}
LTI Systeme sind matematisch mit linearer DGL mit konstanten Koeffizienten beschreibt.
\[
  x = \sum_{k=0}^n a_k y^{(k)} = a_n y^{(n)} + a_{n-1} y^{(n-1)}+ \cdots + a_0 y
\]

\subsection{Impulsantwort}
\subsection{Frequenzgang}
\subsection{Eigenschwingungen}
\subsection{Station\"are L\"osung}


\begin{sidewaystable}
\section{Tabellen}
{
  \renewcommand{\arraystretch}{2}
  \begin{tabularx}{\textwidth}{X *{2}{
      >{\(\displaystyle}r<{\)}
      @{\(\hspace{2mm}\corresponds\hspace{2mm}\)}
      >{\(\displaystyle}l<{\)}}
  }
    Korrespondenzpaar &
      f(t) &
      F(\omega) &
      \sigma(t) f(t) &
      F(s) \\

    Symmetrie &
      F(t) &
      2\pi f(\omega) &
      F(t) &
      \\

    \midrule

    Verschiebung &
      f(t - \tau) &
      F(\omega) \cdot e^{-j\omega\tau} &
      \sigma(t - \tau) f(t - \tau) &
      F(s) \cdot e^{-s\tau} \\

    Modulation / D\"ampfung &
      f(t) \cdot e^{j\Omega t} &
      F(\omega - \Omega) &
      \sigma(t) f(t) \cdot e^{at} &
      F(s - a) \\

    Streckung &
      f(\lambda t) &
      \frac{1}{|\lambda|} F(\frac{\omega}{\lambda}) &
      \sigma(\lambda t) f(\lambda t) &
      \frac{1}{\lambda} F(\frac{s}{\lambda}) \\

    Differentiation &
      \mathcal{D}^k f(t) = f^{(k)}(t) &
      (j\omega)^k F(\omega) &
      \mathcal{D}^k f^{(k)}(t) &
      s^k F(s) - s^{k-1} f(0^+) - \cdots - f^{(k-1)}(0^+) \\

    Integration &
      \mathcal{I}^k f(t) = \int\limits_{-\infty}^{\infty} f(u) \di{u^k} &
      \frac{1}{(j\omega)^k} F(\omega) &
      \mathcal{I} f(t) = \int\limits_0^t f(u) \di{u} &
      \frac{1}{s} F(s) \\

    Hilbert Transform &
      \hilbert f(t) = \frac{1}{\pi} \int\limits_{-\infty}^{\infty} \frac{f(u) \di{u} }{t-u} &
      -j \sgn(\omega) F(\omega) \\

    \midrule

    Faltung &
      g(t) * h(t) &
      G(\omega) \cdot H(\omega) &
      g(t) * h(t) &
      G(s) \cdot H(s) \\

    &
      g(t) \cdot h(t) &
      \frac{1}{2\pi} G(\omega) * H(\omega) &
      g(t) \cdot h(t) &
      \lim_{\gamma\to\infty} \frac{1}{2\pi j}
        \int\limits_{x - j\gamma}^{x + j\gamma} F(u) \cdot G(s - u) \di{u} \\

    \midrule

    Dirac Delta &
      \delta(t) &
      1 &
      \delta(t) &
      1 \\

    Heaviside &
      \sigma(t) &
      \frac{1}{j\omega} &
      \sigma(t) &
      \frac{1}{s} \\

   Sinus &
    \sin (\Omega t) &
    -j\pi\left( \delta(\omega - \Omega) - \delta(\omega + \Omega) \right) &
    \sigma(t)\sin (\omega t) &
    \frac{s}{s^2 + \omega^2} \\

  Cosinus &
    \cos (\Omega t) &
    \pi\left( \delta(\omega - \Omega) + \delta(\omega + \Omega) \right) &
    \sigma(t)\cos (\omega t) &
    \frac{\omega}{s^2 + \omega^2} \\

  Monom &
    t^k &
    &
    \sigma(t) t^k &
    \frac{n!}{s^{n+1}} \\

  \end{tabularx}
}
\end{sidewaystable}


\end{document}
