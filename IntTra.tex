% !TeX program = xelatex
% !TeX encoding = utf8
% !TeX root = IntTra.tex

%% TODO: publish to CTAN
\documentclass[margin=minimal]{tex/hsrzf}

%%%%%%%%%%%%%%%%%%%%%%%%%%%%%%%%%%%%%%%%%%%%%%%%%%%
% Packages

%% TODO: publish to CTAN
\usepackage{tex/hsrstud}
\usepackage{tex/inttra}

%% Language configuration
\usepackage{polyglossia}
\setdefaultlanguage[variant=swiss]{german}

%% License configuration
\usepackage[
    type={CC},
    modifier={by-nc-sa},
    version={4.0},
    lang={german},
]{doclicense}

%% Place figures
\usepackage{float}

%% Nicer tables
\usepackage{tabularx}
\usepackage{array}
\usepackage{booktabs}

%% Pretty drawings and equations
\usepackage{tikz}
\usetikzlibrary{backgrounds}
\usetikzlibrary{calc}

%% Landscape contents
\usepackage{rotating}

%%%%%%%%%%%%%%%%%%%%%%%%%%%%%%%%%%%%%%%%%%%%%%%%%%%
% Metadata

\course{Elektrotechnik}
\module{IntTra}
\semester{Herbstsemester 2020}

\authoremail{naoki.pross@ost.ch}
\author{\textsl{Naoki Pross} -- \texttt{\theauthoremail}}

\title{\texttt{\themodule} Zusammenfassung}
\date{\thesemester}

%%%%%%%%%%%%%%%%%%%%%%%%%%%%%%%%%%%%%%%%%%%%%%%%%%%
% Document

\begin{document}

\maketitle
\tableofcontents
\section*{Lizenz}
\doclicenseThis

\clearpage

\section{Zusammenhang zwischen Integraltransformationen}
\begin{figure}[H] \centering
  
\begin{tikzpicture}[semithick]

  \node[fill=blue!20,
        minimum height=5cm,
        minimum width=5cm] (time) at (0,0) {};

  \node[fill=red!20,
        minimum height=5cm,
        minimum width=5cm,
        anchor=east] (freq) at (-3,0) {};

  \node[fill=magenta!20,
        minimum height=5cm,
        minimum width=4cm,
        anchor=west] (cfreq) at (3,0) {};

  \node[blue!80!black, above] at (time.north) {Zeitbereich};
  \node[red!80!black, above] at (freq.north) {Frequenzbereich};
  \node[magenta!60!black, above] at (cfreq.north) {Bildbereich};



  \node[anchor=south west, rotate=-90, magenta!80!black]
    at (cfreq.north east) {\(z\)-Ebene};
  \node[anchor=south east, rotate=-90, magenta!80!black]
    at (cfreq.south east) {\(s\)-Ebene};

  \node[anchor=south east, rotate=90]
    at (freq.north west) {\textsc{Diskret}};
  \node[anchor=south west, rotate=90]
    at (freq.south west) {\textsc{Kontinuierlich}};

  \node (xt) at ($(time)  - (0,1.75)$) {\(f(t)\)};
  \node (xn) at ($(time)  + (0,1.75)$) {\(y_k\)};

  \node (Xo) at ($(freq)  + (-.75,-1.75)$) {\(F(\omega)\)};
  \node (Xn) at ($(freq)  + (-.75,1.75)$) {\(\hat{c}_k\)};
  \node (cn) at ($(freq)  + (-.75,0)$)    {\(c_k\)};

  \node (Xs) at ($(cfreq) - (0,1.75)$) {\(F(s)\)};
  \node (Xz) at ($(cfreq) + (0,1.75)$) {\(F(z)\)};

  \draw[->] (xt) to node[midway, right] {\(\Sh_{\Delta t}\)} (xn);

  \draw[<-] (Xs) to node[near start, right]
    {\(\displaystyle s = \frac{\ln(z)}{T}\)} (Xz);

  \draw[<->] (cn) to
    node[near end, left, anchor=east] {\(\displaystyle\lim_{T \to \infty}\)}
    node[near start, right] {\(F(k\omega_0) = 2Tc_k\)} (Xo);

  \draw[->] (Xn) to node[midway, left, anchor=east]
    {\(\displaystyle 2\hat{\omega} \leq \frac{2\pi}{\Delta t}\)} (cn);

  \draw[<->] (xt) to[out=135, in=10]
    node[near end, sloped, above] {\(\mathcal{F}_T\)}
    node[near start, sloped, above] {\(\mathcal{F}_T^{-1}\)} (cn);

  \draw[<->] (xt) to
    node[near end, above] {\(\mathcal{F}\)}
    node[near start, above] {\(\mathcal{F}^{-1}\)} (Xo);

  \draw[<->] (xt) to
    node[near end, above] {\(\mathcal{L}\)}
    node[near start, above] {\(\mathcal{L}^{-1}\)} (Xs);

  \draw[<->] (xn) to
    node[near end, above] {\(\mathtt{DFT}_N\)}
    node[near end, below] {\(\mathtt{FFT}_N\)}
    node[near start, above] {\(\mathtt{IDFT}_N\)}
    node[near start, below] {\(\mathtt{IFFT}_N\)} (Xn);

  \draw[<->] (xn) to
    node[near end, above] {\(\mathcal{Z}\)}
    node[near start, above] {\(\mathcal{Z}^{-1}\)} (Xz);

  \draw[->] (Xs) .. controls ($(Xs)-(0,2)$) and ($(Xo)-(0,2)$) ..
    node[midway, above] {\(s = j\omega\)} (Xo);
\end{tikzpicture}

\end{figure}

\section{Fourier}
\subsection{Fourier Reihe und Transformation}

% \begin{center}
%   \begin{tabular}{>{\(\displaystyle}l<{\)} >{\(\displaystyle}l<{\)}}
%     \text{Fourier Reihe} & \text{Fourier Transform} \\
%     \midrule
%     c_k       = X(k\omega_0) = \frac{1}{T} \int\limits_{-T/2}^{T/2} x(t)e^{-jk\omega_0 t} \di{t} &
%     X(\omega) = \mathcal{F}x(t) = \int\limits_{-\infty}^{\infty} x(t)e^{-j\omega t} \di{t} \\
%     x(t)      = \sum_{k=-\infty}^{\infty} c_k e^{jk\omega_0 t} &
%     x(t)      = \frac{1}{2\pi} \int\limits_{-\infty}^{\infty} X(\omega) e^{j\omega t} \di{\omega}
%   \end{tabular}
% \end{center}


\subsection{Diskrete und schnelle Fouriertransformation}
\subsection{Hilbertstransformation}

\section{Laplace Transformation}
\[
  F(s) = \laplace f(t) = \int\limits_{0}^{\infty} f(t) e^{-st} \di{t}
  \qquad
  f(t) = \ilaplace F(s) = \lim_{\gamma\to\infty} \frac{1}{2\pi j} 
  	\int\limits_{x - j\gamma}^{x + j\gamma} F(s)\, e^{st} \di{s}
\]
\subsection{Konvergenzbereich}
Zuerst kann man klar sehen, dass ausser wenn 
\(\mathcal{O}(f) \geq \mathcal{O}(e^{\alpha t})\)
konvergiert das Laplaceintegral f\"ur \(\real(s) < 0\) nicht.
Allgemeiner die Bedingung ist \(\real(s) < a \in \mathbf{R}\cup\{\pm\infty\}\).

\section{Lineare Differenzialgleichungen}
\begin{figure}[H] \centering
  \begin{tikzpicture}[semithick]
  \node[fill=blue!20,
        minimum height=3cm,
        minimum width=3.5cm,
        anchor=east] (time) at (-.5, 0) {};

  \node[fill=green!20,
        minimum height=3cm,
        minimum width=5cm,
        anchor=west] (image) at (.5, 0) {};

  \node[minimum height=3.25cm,
        minimum width=1.5cm] (R) {};

  \node[above, color=blue!70!black] at (time.north)
    {Zeitbereich};

  \node[above, color=green!60!black] at (image.north)
    {Bildbereich};

  \node[anchor=north east, yshift=-5mm] (time-dgl) at (R.north west)
    {\(\sum_k a_k y^{(k)} = f(t)\)};

  \node[anchor=north west, yshift=-5mm] (image-dgl) at (R.north east)
    {\(p(s) Y(s) - h(s) = F(s) \)};

  \node[anchor=south west, yshift=5mm] (image-sol) at (R.south east)
    {\(Y(s) = G(s) \cdot \big( F(s) + h(s) \big)\)};

  \node[anchor=south east, yshift=5mm] (time-sol) at (R.south west)
    {\(y(t)\)};

  \draw[->] (time-dgl) to
    node[midway, above] {\(\mathcal{L}\)} (image-dgl);

  \draw[->] (image-dgl) to
    node[midway, right] {\(G(s) = 1/p(s)\)} (image-dgl |- image-sol.north);

  \draw[->] (image-sol) to
    node[midway, above] {\(\mathcal{L}^{-1}\)} (time-sol);

\end{tikzpicture}

\end{figure}

\begin{figure}[H] \centering
  \begin{tikzpicture}[
    semithick,
    mtext/.style = {outer sep=2mm},
  ]

  \matrix[row sep=1.25cm, column sep=1cm] (M) {
    \node[mtext] (time-dgl) {\(\sum_k a_k y^{(k)} = f\)}; &&
      \node[mtext] (image-dgl) {\(pY - h = F\)}; &
      \node (A) {}; \\

    &&
      \node[mtext] (image-dgl-g) {\(Y_G = G\cdot F\)}; &
      \node[mtext] (image-dgl-h) {\(Y_H = h/p\)}; \\

    \node[mtext] (time-sol) {\(y\)}; &&
      \node[mtext] (image-sol) {\(Y = Y_G + Y_H\)}; &
      \node (B) {}; \\
  };

  \draw[->] (time-dgl) to node[midway, above] {\(\mathcal{L}\)} (image-dgl);

  \draw[->] (image-dgl) .. controls (A) ..
    node[near end, right] {\(F(s) = 0\)} (image-dgl-h);

  \draw[->] (image-dgl) to
    node[midway, right] {\(h(s) = 0\)} (image-dgl-g);

  \draw[->] (image-dgl-g) to (image-sol);
  \draw[->] (image-dgl-h) .. controls (B) .. (image-sol);

  \draw[->] (image-sol) to node[pos=.3, above] {\(\mathcal{L}^{-1}\)} (time-sol);

  \draw[->, dashed] (time-dgl) to (time-sol);

  \begin{pgfonlayer}{background}
    \node (T1) at ($(time-dgl.north west) + (-.5, .3)$) {};
    \node (T2) at ($(time-dgl.east |- time-sol.south east) + (.5, -.3)$) {};

    \node (I1) at ($(image-dgl.north west) + (-.5,0)$) {};
    \node (I2) at ($(B.south east) + (1.7,0)$) {};

    % adjust heights of I to match T
    \node (I1) at (T1 -| I1) {};
    \node (I2) at (T2 -| I2) {};

    \node[color=blue!70!black, above right] at (T1) {Zeitbereich};
    \node[color=magenta!70!black, above right] at (I1) {Bildbereich};

    \fill[color=blue!20] (T1) rectangle (T2);
    \fill[color=magenta!20] (I1) rectangle (I2);
  \end{pgfonlayer}
\end{tikzpicture}

\end{figure}
\section{Lineare Zeitinvariante Systeme}
\begin{figure}[H] \centering
  \begin{tikzpicture}[
    system/.style = {draw, thick, inner sep = 4mm, outer sep = 1mm}
  ]
  \matrix[row sep=3mm, column sep=1.5cm] (M) {
    \node (x) {\(x(t)\)}; &
    \node (g) {\(g(t) = y_\delta (t)\)}; &
    \node (y) {\(y(t) = g(t) * x(t)\)}; \\

    &
    \node (h) {\(h(t)\)}; &
    \node (yw) {\(y_\omega(t) = h(t) * x(t)\)}; \\

    \node (in) {Anregung}; &
    \node[system, fill=white] (sys) {LTI-System \(\mathcal{S}\)}; &
    \node (out) {Antwort}; \\

    \node (X) {\(X(s)\)}; &
    \node (G) {\(G(s) = 1/p(s)\)}; &
    \node (Y) {\(Y(s) = G(s) \cdot X(s)\)}; \\

    \node (Xw) {\(X(\omega)\)}; &
    \node (H)  {\(H(\omega) = G(j\omega)\)}; &
    \node (Yw) {\(Y_\omega (\omega) = H(\omega) \cdot X(\omega)\)}; \\
  };

  \draw[thick, ->] (in) to (sys);
  \draw[thick, ->] (sys) to (out);

  \begin{pgfonlayer}{background}
    \coordinate (T1) at ($(x.north west) - (.8,-.1)$);
    \coordinate (T2) at ($(yw.south east) + (.8,-.1)$);

    \coordinate (B1) at ($(X.north west) - (0,-.1)$);
    \coordinate (B2) at ($(Y.south east) + (0,-.1)$);

    \coordinate (F1) at ($(Xw.north west) - (0,-.1)$);
    \coordinate (F2) at ($(Yw.south east) + (0,-.1)$);

    \fill[color=blue!20] (T1) rectangle (T2);
    \fill[color=magenta!20] (B1 -| T1) rectangle (B2 -| T2);
    \fill[color=red!20] (F1 -| T1) rectangle (F2 -| T2);
    % \fill[top color=blue!20, bottom color=magenta!20]
      % (T1) rectangle (B2);
  \end{pgfonlayer}
\end{tikzpicture}
\end{figure}
Ein System mit der folgenden Eigenschaften ist als LTI bezeichnet.
\begin{center}
  \begin{tabularx}{\linewidth}{l >{\(\displaystyle }X<{\)}}
    Linear & \mathcal{S} (ax_1 + bx_2) = a\mathcal{S} x_1 + b\mathcal{S} x_2 = ay_1 + by_2 \\
    Zeitinvariant & \mathcal{S} x(t + t_0) = y(t + t_0) \\
  \end{tabularx}
\end{center}
LTI Systeme sind matematisch mit linearer DGL mit konstanten Koeffizienten beschreibt.

\subsection{Impulsantwort}
Die \emph{Impulsantwort} \(y_\delta\) ist die Antwort des Systems zur Anregung mit \(\delta(t)\).

\subsection{Frequenzgang}

\subsection{Eigenschwingungen}
Die \emph{Eigenschwingung} ist die Antwort des Systems \emph{ohne} Anregung (\(x(t) = 0\)),
d. h. die homogene L\"oesung von der DGL.
Sie ist die Reaktion des Systems auf gegebene Anfangsbedingungen.
Seien die Anfangsbedingungen \(y(0) = c_0, \dots, y^{(n-1)}(0) = c_{n-1}\)
\[
  0 = \sum_{k=0}^n a_k y^{(k)}
  \quad\stackrel{\laplace}{\implies}\quad
  0 = \underbrace{(a_n s^n + \cdots + a_0)}_{p(s)} Y(s) 
    - \underbrace{(s^{n-1} c_0 + \cdots + c_{n-1})}_{r(s)}
  \quad\iff\quad
  Y(s) = \frac{r(s)}{p(s)}
\]

\subsection{Station\"are L\"osung}


\section{Tabellen}
\begin{sidewaystable}
{
  \renewcommand{\arraystretch}{2}
  \begin{tabularx}{\textwidth}{X *{2}{
      >{\(\displaystyle}r<{\)}
      @{\(\hspace{2mm}\corresponds\hspace{2mm}\)}
      >{\(\displaystyle}l<{\)}}
  }
    Korrespondenzpaar &
      f(t) &
      F(\omega) &
      \sigma(t) f(t) &
      F(s) \\

    Symmetrie &
      F(t) &
      2\pi f(\omega) &
      F(t) &
      \\

    \midrule

    Verschiebung &
      f(t - \tau) &
      F(\omega) \cdot e^{-j\omega\tau} &
      \sigma(t - \tau) f(t - \tau) &
      F(s) \cdot e^{-s\tau} \\

    Modulation / D\"ampfung &
      f(t) \cdot e^{j\Omega t} &
      F(\omega - \Omega) &
      \sigma(t) f(t) \cdot e^{at} &
      F(s - a) \\

    Streckung &
      f(\lambda t) &
      \frac{1}{|\lambda|} F(\frac{\omega}{\lambda}) &
      \sigma(\lambda t) f(\lambda t) &
      \frac{1}{\lambda} F(\frac{s}{\lambda}) \\

    Differentiation &
      \mathcal{D}^k f(t) = f^{(k)}(t) &
      (j\omega)^k F(\omega) &
      \mathcal{D}^k f^{(k)}(t) &
      s^k F(s) - s^{k-1} f(0^+) - \cdots - f^{(k-1)}(0^+) \\

    Integration &
      \mathcal{I}^k f(t) = \int\limits_{-\infty}^{\infty} f(u) \di{u^k} &
      \frac{1}{(j\omega)^k} F(\omega) &
      \mathcal{I} f(t) = \int\limits_0^t f(u) \di{u} &
      \frac{1}{s} F(s) \\

    Hilbert Transform &
      \hilbert f(t) = \frac{1}{\pi} \int\limits_{-\infty}^{\infty} \frac{f(u) \di{u} }{t-u} &
      -j \sgn(\omega) F(\omega) \\

    \midrule

    Faltung &
      g(t) * h(t) &
      G(\omega) \cdot H(\omega) &
      g(t) * h(t) &
      G(s) \cdot H(s) \\

    &
      g(t) \cdot h(t) &
      \frac{1}{2\pi} G(\omega) * H(\omega) &
      g(t) \cdot h(t) &
      \lim_{\gamma\to\infty} \frac{1}{2\pi j}
        \int\limits_{x - j\gamma}^{x + j\gamma} F(u) \cdot G(s - u) \di{u} \\

    \midrule

    Dirac Delta &
      \delta(t) &
      1 &
      \delta(t) &
      1 \\

    Heaviside &
      \sigma(t) &
      \frac{1}{j\omega} &
      \sigma(t) &
      \frac{1}{s} \\

   Sinus &
    \sin (\Omega t) &
    -j\pi\left( \delta(\omega - \Omega) - \delta(\omega + \Omega) \right) &
    \sigma(t)\sin (\omega t) &
    \frac{s}{s^2 + \omega^2} \\

  Cosinus &
    \cos (\Omega t) &
    \pi\left( \delta(\omega - \Omega) + \delta(\omega + \Omega) \right) &
    \sigma(t)\cos (\omega t) &
    \frac{\omega}{s^2 + \omega^2} \\

  Monom &
    t^k &
    &
    \sigma(t) t^k &
    \frac{n!}{s^{n+1}} \\

  \end{tabularx}
}
\end{sidewaystable}


\end{document}
