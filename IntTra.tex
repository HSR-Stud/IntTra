 %%%%%%%%%%%%%%%%%%%%%%%%%
% Dokumentinformationen %
%%%%%%%%%%%%%%%%%%%%%%%%%
\newcommand{\titleinfo}{Integraltransformationen - Zusammenfassung}
\newcommand{\authorinfo}{Dev4Future, sowie weitere Autoren}
\newcommand{\versioninfo}{HS / 2019}

%weitere Autoren
%Dev4Future, Braun \& Co, J\"urg \& Co, Hannes, K\"orner, Gwerder, Niedermann

%%%%%%%%%%%%%%%%%%%%%%%%%%%%%%%%%%%%%%%%%%%%%
% Standard projektübergreifender Header für
% - Makros 
% - Farben
% - Mathematische Operatoren
%
% DORT NUR ERGÄNZEN, NICHTS LÖSCHEN
%%%%%%%%%%%%%%%%%%%%%%%%%%%%%%%%%%%%%%%%%%%%%

% Genereller Header
\documentclass[10pt,twoside,a4paper,fleqn]{article}
\usepackage[utf8]{inputenc}
\usepackage[left=1cm,right=1cm,top=1cm,bottom=1cm,includeheadfoot]{geometry}
\usepackage[ngerman]{babel,varioref}

% Pakete
\usepackage{amssymb,amsmath,fancybox,graphicx,lastpage,wrapfig,fancyhdr,hyperref,verbatim}

%%%%%%%%%%%%%%%%%%%%%%%%%%%%
% Mathematische Operatoren %
%%%%%%%%%%%%%%%%%%%%%%%%%%%%
\DeclareMathOperator{\Real}{Re}
\DeclareMathOperator{\Imag}{Im}



% Fouriertransformationen
\unitlength1cm
% Zeitbereich -- Frequenzbereich
\newcommand{\laplace}
{
\begin{picture}(1,0.5)
\put(0.2,0.1){\circle{0.14}}\put(0.27,0.1){\line(1,0){0.5}}\put(0.77,0.1){\circle*{0.14}}
\end{picture}
}


% Frequenzbereich -- Zeitbereich
\newcommand{\Laplace}
{
\begin{picture}(1,0.5)
\put(0.2,0.1){\circle*{0.14}}\put(0.27,0.1){\line(1,0){0.45}}\put(0.77,0.1){\circle{0.14}}
\end{picture}
}



%%%%%%%%%%%%%%%%%%%%%%%%%%%%
% Allgemeine Einstellungen %
%%%%%%%%%%%%%%%%%%%%%%%%%%%%
%pdf info
\hypersetup{
	pdfauthor={\authorinfo},
	pdftitle={\titleinfo},
	colorlinks=false}
\author{\authorinfo}
\title{\titleinfo}

%Kopf- und Fusszeile
\pagestyle{fancy}
\fancyhf{}
%Linien oben und unten
\renewcommand{\headrulewidth}{0.5pt} 
\renewcommand{\footrulewidth}{0.5pt}

\fancyhead[L]{\titleinfo{ }\tiny{(\versioninfo)}}
%Kopfzeile rechts bzw. aussen
\fancyhead[R]{Seite \thepage { }von \pageref{LastPage}}
%Fusszeile links bzw. innen
\fancyfoot[L]{\footnotesize{\authorinfo}}
%Fusszeile rechts bzw. ausen
\fancyfoot[R]{\footnotesize{\today}}

% Einrücken verhindern versuchen
\setlength{\parindent}{0pt} %\tabcolsep


\usepackage{tikz}

% Möglichst keine Ergänzungen hier, sondern in header.tex
\begin{document}
%%%%%%%%%%%%%%%%%%%%%%%%%%%%%%%%%%%%%%%%%%%%%%%%%%%%%%%%%%%%%%%%%%%%%%%%%%%%%%%%%%%%%%%%%%%%%%%%
%%%%%%%%%%%%%%%%%%%%%%%%%%%%%%%%%%%%%%%%%%%%%%%%%%%%%%%%%%%%%%%%%%%%%%%%%%%%%%%%%%%%%%%%%%%%%%%%

% Einleitung

\section{Varianten der Integraltransformationen}
\begin{tabular}{|l||l|l|}
\hline
\textbf{Signalart}
	& diskret
	& kontinuierlich \\
\hline \hline
periodisch
	& Diskrete Fourier-Transformation
	& Fourierreihe \\
\hline
impulsförmig
	& ``Fourierreihe mit $T = Impulsdauer$''
	& Fourierintegral \\
\hline
kausal
	& Z-Transformation
	& Laplace-Transformation \\
\hline
\end{tabular}

% Signale und Systeme
\section{Signale und Systeme}
	\begin{tabular}{|l|l|}
    	\hline
    	\textbf{Linearität} & \textbf{Zeitinvarianz}\\
    	\hline
    	$S(x1+x2)=S(x1)+S(x2)$ & $S(x(t-t_0)=S(x)\cdot x(t-t_0)$ \\
    	$S(c\cdot x)=c\cdot S(x)$ & \\
		\hline    
    \end{tabular}
  	
	\subsection{Lineare Systeme}
		\textbf{Basissignale}
		\begin{list}{$\bullet$}{\setlength{\itemsep}{0cm} \setlength{\parsep}{0cm} \setlength{\topsep}{0cm}} 
          \item Lineare Systeme sind durch die Antworten auf die
          Basissignale bestimmt.
          \item Basissignale müssen linear unabhängig voneinander sein, d.h. ein
		Basissignal darf nicht durch \textbf{Linearkombination} anderer Basissignale
		darstellbar sein          
		  \item Alle möglichen Eingangs-Funktionen müssen durch eine Linearkombination der
		Basissignale dargestellt werden können. $\Rightarrow$ \textbf{Periode des Eingangssignals =	Anzahl Basissignale}
        \end{list}
        \vspace{.2cm}
		\textbf{Berechnung der Systemantwort aufgrund der Basissignale und der
		Anregung}\\
		1. Eingangssignal $x$ als Linearkombination der Basisvektoren darstellen
		$\Rightarrow$ lineares Gleichungssystem\\
		$\Rightarrow x=r\cdot a + s\cdot b + t\cdot c\qquad$ ($x=$
		Eingangssignal; $a,b,c=$ Basisvektoren; $r,s,t=$
		Linearkombinationsparameter)\\ 
		2. Systemantwort $y=r\cdot S(a) + s\cdot S(b) + t\cdot S(c); \qquad (S(a)=$
		Systemantwort der Basis $a$)
	
	\subsection{Lineare zeitinvariante Systeme (LTI-Systeme)}
		LTI-Systeme sind durch ihre Impulsantwort $h$ vollständig bestimmt\\ \\
		\textbf{Berechnung der Systemantwort von diskreten LTI-Systemen}\\
		$\; y=x*h \qquad$ ($y=$ Systemantwort; $x=$ Eingangssignal; $h=$
		Impulsantwort)\\
		
		\textbf{Berechnung der Systemantwort von kontinuierlichen LTI-Systemen}\\
		\begin{tabular}{ll}
			\parbox{8cm}{
			$$s_2(t) = h(t) * s_1(t) \laplace S_2(s) = H(s) S_1(s)$$
			$$h(t) \laplace H(s)$$}
			& \parbox{4cm}{
			\includegraphics[width=5cm]{./bilder/utf-theorie.png}}\\
		\end{tabular}	
		
	\subsection{Faltung}
	$y(t) = f(t)\ast g(t) = g(t) \ast f(t) = (f \ast g)(t) :=
	\int\limits_{-\infty}^\infty f(u) \cdot g(t-u)\,du =
	\int\limits_{-\infty}^\infty f(t-u) \cdot h(u)\,du $ \\
	wobei gilt: $h\left/t\right) g(t) = $ Impulsantwort des Systems \\
	Hat $g\left(t\right)$ keine negative Argumente dann gilt :
	$\left(g \ast t \right)\left(t\right)=\int\limits_{-\infty}^t f(u) \cdot
	g(t-u)\,du$\\
	Hat $f\left(t\right)$ keine negative Argumente (Einschaltvorgang) dann gilt :
	$\left(g \ast t \right)\left(t\right)=\int\limits_{0}^t f(u) \cdot
	g(t-u)\,du$\\
	Bei einer Faltung mit einer $\delta\left(t\right)$Funktion
	gilt:$f\left(t\right) \ast \delta\left(t\right) = f\left(t\right)$\\
	
	\newpage
	
	\begin{multicols}{2}
		\textbf{grafische Interpretation:}
		\begin{enumerate}
  			\item einfacheres Signal an der Y-Achse spiegeln
  			\item Verschiebung um t nach rechts
  			\item Multiplikation und Integration der beiden Signale
		\end{enumerate}
		\columnbreak
		\textbf{Bestimmung der Grenzen:}
		\begin{enumerate}
		  \item Koordinatensystem: X-Achse: t, Y-Achse: u
		  \item Erster Faktor $\rightarrow$ Streifen parallel zur X-Achse
		  \item Zweiter Faktor $\rightarrow$ Streiffen parallel zur $45^{\circ}$-Geraden
		  \item Grenzen für ein bestimmtes t ablesen
		\end{enumerate}
	\end{multicols}	

% Diskrete Fourier Transformation
%\section{Diskrete Fourier Transformation (DFT)}
	$$\boxed{s(h)=\sum_{k=0}^{N-1}\hat c_k e^{jhk\frac{2\pi}{N}}=\sum_{k=0}^{N-1}
	\left[ \hat{a}_k \cos\left(hk \frac{2 \pi}{N}\right)+\hat{b}_k \sin\left(hk
	\frac{2 \pi}{N}\right) \right]} \qquad N=\text{Periodenanzahl}$$\\
	$$\hat{c}_k=\frac{1}{N}\sum_{h=0}^{N-1}s(h)
	e^{-jhk\frac{2\pi}{N}}=\hat{a}_k-j\hat{b}_k \qquad \hat{a}_k=\frac{1}{N}
	\sum_{h=0}^{N-1}s(h) \cos\left(hk \frac{2 \pi}{N}\right)=Re(\hat{c}_k) \qquad
	\hat{b}_k=\frac{1}{N} \sum_{h=0}^{N-1}s(h) \sin\left(hk \frac{2
	\pi}{N}\right)=-Im(\hat{c}_k)$$\\	

	\subsection{Berechnung mit Matrizen}
		\textbf{Transformation}\\
		1. Periode $N$ des Signalvektors $\vec{s}=
		\begin{bmatrix}
		s(0) \\
		s(1) \\
		s(N-1)\\
		\end{bmatrix}$ bestimmen\\ \\
		2. Einheitswurzel $w$ berechnen: $w=e^{j\frac{2 \pi}{N}}$\\ \\
		3. Matrix $W$ berechnen: $W=
		\begin{bmatrix}
		w^0 & w^0 & w^0 & \ldots & w^0\\
		w^0 & w^1 & w^2 & \ldots & w^{N-1}\\
		w^0 & w^2 & w^4 & \ldots & w^{2(N-1)}\\
		\ldots & \ldots & \ldots & \ldots & \ldots\\
		w^0 & w^{N-1} & w^{2(N-1)} & \ldots & w^{(N-1)(N-1)}                        
		\end{bmatrix}$\\ \\
		4. Matrix $V$ berechnen: $V=W^*$ (konj. komplex)\\ \\
		5. Koeffizienten bzw. Fouriertransformierte $\vec{c}$ berechnen:
		$\vec{c}=\frac{1}{N}V\vec{s}$\\

		\begin{minipage}{13cm}
			\textbf{Rücktransformation}\\
			1. Matrix $W$ (wie bei der Transformation beschrieben) berechnen \\
			2. Signalvektor $\vec{s}$ berechnen: $\vec{s}=W\vec{c}$	
			\subsection{Matrizenmultiplikation}
			\begin{tabular}{ll}
				$\frac14
				\begin{bmatrix}
				    6 & -1 & 4 \\
				    3 & 2 & -2 \\
				    0 & -3 & -1
				\end{bmatrix}
				\cdot
				\begin{bmatrix}
					1 \\
				    4 \\
				    3 
				\end{bmatrix}
				=
				\frac14
				\begin{bmatrix}
					6 \cdot 1 + (-1) \cdot 4 + 4 \cdot 3\\
					3 \cdot 1 + 2 \cdot  4 + (-2) \cdot 3\\
					0 \cdot 1 + (-3) \cdot 4 + (-1) \cdot 3  
				\end{bmatrix}
				=
				\frac14
				\begin{bmatrix}
				    14\\
				    5\\
				    -15
				\end{bmatrix}
				=
				\begin{bmatrix}
		        	3.5\\
		        	1.25\\
		        	-3.75
		        \end{bmatrix}$
		    \end{tabular}		
        \end{minipage}
		\begin{minipage}[c]{5cm}
        	\includegraphics[width=5cm]{../IntTra/bilder/matrix.png}
        \end{minipage}
		
	\subsection{Einige W-Matrizen}
		\begin{tabular}{l l l l}
        N = 2 & N = 3 & N = 4 & N = 6\\
		$\begin{bmatrix}
		1 & 1\\
		1 & -1\\              
		\end{bmatrix}$ &
		$\begin{bmatrix}
		1 & 1 & 1\\
		1 & -\frac{1}{2}+j\frac{\sqrt{3}}{2} & -\frac{1}{2}-j\frac{\sqrt{3}}{2}\\
		1 & -\frac{1}{2}-j\frac{\sqrt{3}}{2} & -\frac{1}{2}+j\frac{\sqrt{3}}{2}\\
		\end{bmatrix}$ &
		$\begin{bmatrix}
		1 & 1 & 1 & 1 \\
		1 & j & -1 & -j\\
		1 & -1 & 1 & -1\\
		1 & -j & -1 & j\\                   
		\end{bmatrix}$ &
		$\begin{bmatrix}
		1 & 1 & 1 & 1 & 1 & 1\\
		1 & \frac{1}{2}+j\frac{\sqrt{3}}{2} & -\frac{1}{2}+j\frac{\sqrt{3}}{2} & -1
		& -\frac{1}{2}-j\frac{\sqrt{3}}{2} & \frac{1}{2}-j\frac{\sqrt{3}}{2}\\
		1 & -\frac{1}{2}+j\frac{\sqrt{3}}{2} & -\frac{1}{2}-j\frac{\sqrt{3}}{2} & 1
		& -\frac{1}{2}+j\frac{\sqrt{3}}{2} & -\frac{1}{2}-j\frac{\sqrt{3}}{2}\\
		1 & -1 & 1 & -1 & 1 & -1\\
		1 & -\frac{1}{2}-j\frac{\sqrt{3}}{2} & -\frac{1}{2}+j\frac{\sqrt{3}}{2} & 1
		& -\frac{1}{2}-j\frac{\sqrt{3}}{2} & -\frac{1}{2}+j\frac{\sqrt{3}}{2}\\ 
		1 & \frac{1}{2}-j\frac{\sqrt{3}}{2} & -\frac{1}{2}-j\frac{\sqrt{3}}{2} & -1
		& -\frac{1}{2}+j\frac{\sqrt{3}}{2} & \frac{1}{2}+j\frac{\sqrt{3}}{2}\\ 
		\end{bmatrix}$
		\end{tabular}

		\begin{tabular}{l }
        N = 8\\
 		$\begin{bmatrix}
		1 & 1 & 1 & 1 & 1 & 1 & 1 & 1\\ 
		1 & \frac{\sqrt{2}}{2}+\frac{\sqrt{2}}{2}j & j &
		-\frac{\sqrt{2}}{2}+\frac{\sqrt{2}}{2}j & -1 & 
		-\frac{\sqrt{2}}{2}-\frac{\sqrt{2}}{2}j & -j & 
		\frac{\sqrt{2}}{2}-\frac{\sqrt{2}}{2}j\\
		1 & j & -1 & -j & 1 & j & -1 & -j\\
		1 &	-\frac{\sqrt{2}}{2}+\frac{\sqrt{2}}{2}j & -j & 
		\frac{\sqrt{2}}{2}+\frac{\sqrt{2}}{2}j & -1 & 
		\frac{\sqrt{2}}{2}-\frac{\sqrt{2}}{2}j & j &
		-\frac{\sqrt{2}}{2}-\frac{\sqrt{2}}{2}j\\
		1 & -1 & 1 & -1 & 1 & -1 & 1 & -1\\
		1 &	-\frac{\sqrt{2}}{2}-\frac{\sqrt{2}}{2}j & j &
		\frac{\sqrt{2}}{2}-\frac{\sqrt{2}}{2}j & -1 &
		\frac{\sqrt{2}}{2}+\frac{\sqrt{2}}{2}j & -j &
		-\frac{\sqrt{2}}{2}+\frac{\sqrt{2}}{2}j\\
		1 & -j & -1 & j & 1 & -j & -1 & j\\
		1 &	\frac{\sqrt{2}}{2}-\frac{\sqrt{2}}{2}j & -j & 
		-\frac{\sqrt{2}}{2}-\frac{\sqrt{2}}{2}j & -1 &
		-\frac{\sqrt{2}}{2}+\frac{\sqrt{2}}{2}j & j &
		\frac{\sqrt{2}}{2}+\frac{\sqrt{2}}{2}j
		\end{bmatrix}$
		\end{tabular}


	
	
	
	
%\newpage

%Funktionen
\section{Funktionen}
%	\begin{table}[h]
%		\centering
%		\resizebox{\textwidth}{!}{%
%			\begin{tabularx}{1\textwidth}{p{0.25\textwidth}ll}
%				\multicolumn{3}{l}{Sprungfunktion, Schrittfunktion, unit step} \\
%				\includegraphics[width=\textwidth]{./bilder/sprungF.png}      & x      & x      \\ \hline
%				\multicolumn{3}{l}{step} \\
%				&        &        \\ 
%			\end{tabularx}%
%		}
%	\end{table}
	
	\subsection{Impulsfunktion - dirac delta function}
		\begin{minipage}{0.2\textwidth}
			\includegraphics[width=\textwidth]{./bilder/funktionen/diracimpulse.png}
		\end{minipage}
		\qquad
		\begin{minipage}{0.45\textwidth}
			$\delta (t)=\begin{cases} 0 & t\ne 0\\\infty & t=0\end{cases} \qquad
			\text{und} \qquad \int\limits_{-\infty}^\infty \delta(t) \, \mathrm dt = 1 $\\
		\end{minipage}
		\qquad
		\begin{minipage}{0.25\textwidth}						
				$\int\limits_{-\infty}^{\infty}\delta(t-t_0)f(t)dt=f(t_0)$\\
	$\int\limits_{-\infty}^{\infty}\delta(at)\cdot f(t) dt = \frac{1}{|a|} \cdot f(0)$\\
	$s(t) \cdot \delta(t-t_0) = s(t_0)\cdot \delta(t-t_0)$\\
	Fouriertransformierte von $\delta(t)$:\\
	$\delta(t) \; \laplace \; 1(\omega)$ \hspace{0.5cm}
	$\delta(t-t_0) \; \laplace \; e^{-j\omega t_0}$ \hspace{0.5cm}
	$1(t) \; \laplace \; 2\pi \delta(\omega)$
		\end{minipage}
	
	\subsubsection{d-Funktion}
		\includegraphics[width=0.8\textwidth]{./bilder/funktionen/impulsF.png}


	\subsection{Schrittfunktion - unit step}
		\begin{minipage}{0.2\textwidth}
			\includegraphics[width=\textwidth]{./bilder/funktionen/sprungF.png}
		\end{minipage}
		\qquad
		\begin{minipage}{0.45\textwidth}
			$u(t) = \sigma(t) =	\begin{cases}
			0 & \text{f\"ur } t < 0 \\
			\frac{1}{2} \text{(praxis)}  \text{ oder undef. (math.)} & \text{f\"ur } t = 0 \\
			1 & \text{f\"ur } t > 0
			\end{cases}$
		\end{minipage}
		\qquad
		\begin{minipage}{0.25\textwidth}						
			$\sigma(t) \; \laplace \; \frac{1}{j\omega} + \pi\delta(\omega) = \Sigma(\omega)$ \\
			\\
			$\frac{du(t)}{dt}=\delta(t)$\\
		\end{minipage}
	
	\subsection{Signumfunktion}
		\begin{minipage}{0.2\textwidth}
			\includegraphics[width=\textwidth]{./bilder/funktionen/signF.png}
		\end{minipage}
		\qquad
		\begin{minipage}{0.45\textwidth}
			$sgn(t) = \begin{cases} 1 & \text{falls }t > 0 \\ -1 & \text{falls }t < 0 \end{cases}$
		\end{minipage}
		\qquad
		\begin{minipage}{0.25\textwidth}						
			$sgn(t) \; \laplace \; \frac{2}{j\omega}$\\
			\\
			$\frac{1}{\pi t} \; \laplace \; -j sgn(\omega)$\\
		\end{minipage}		

		
	\subsection{Rechteckimpuls}
		\begin{minipage}{0.2\textwidth}
			\includegraphics[width=\textwidth]{./bilder/funktionen/rechteckF.png}
		\end{minipage}
		\qquad
		\begin{minipage}{0.45\textwidth}
			$p_{a}(t)=\begin{cases}
							1 & |t|<a \\ 
							\frac{1}{2} & |t|=a \\ 
							0 & |t|>a
						\end{cases}$
		\end{minipage}
		\qquad
		\begin{minipage}{0.25\textwidth}						
			$sgn(t) \; \laplace \; \frac{2}{j\omega}$\\
			\\
			$\frac{1}{\pi t} \; \laplace \; -j sgn(\omega)$\\
		\end{minipage}
			\begin{tabular}{p{9cm} p{9cm}}
				$r_T(t) \; \laplace \; \frac{2}{\omega} \cdot \sin(\omega T) \Rightarrow$ sinc-Funktion &
				$\int\limits_{-\infty}^{\infty} \frac{\sin(a \omega)}{\omega} d\omega = 
				\begin{cases} \pi & \text{falls }a > 0 \\ -\pi & \text{falls }a < 0 \end{cases}$
			\end{tabular}
		
	\subsection{Dreieckimpuls}
	
	\subsection{Sinc-Funktion}
	
	\subsection{Funktionen manipulieren}
		\includegraphics[width=\textwidth]{./bilder/SignalManip.png}

% Fourierreihe
\section{Fourierreihe}
  	$$\boxed{f(t) = \sum\limits_{k = -\infty}^{\infty} c_k \cdot e^{j k \omega_1
  	t}}= \boxed{\sum\limits_{k = 0}^{\infty} \left(c_k \cdot e^{j k \omega_1
  	t} + \overline{c_k} \cdot e^{-j k \omega_1t}\right)}$$
  	$$\boxed{f(t) = \frac{a_0}{2} + \sum\limits_{k=1}^{\infty} \left[a_k \cos(k
  	\omega_1 t) + b_k \sin(k \omega_1 t)\right]}=\boxed{\frac{A_0}{2} +
  	\sum\limits_{k=1}^{\infty} A_k \cos(k \omega_1 t + \varphi_k)} \quad k\in
  	\mathbb{Z}, \quad \boxed{\omega_1=\frac{2 \pi}{T}=2 \pi f}$$	
	$$\boxed{c_k=\overline{c_{-k}}=\frac{1}{T}\int_0^T{f(t)\cdot
	e^{-jk\omega_1
	t}dt} \; ; \; c_0 = \frac{a_0}{2}} \qquad \boxed{a_0 = \frac{2}{T}\int\limits_0^{T}
	f(t)dt, \quad a_k = \frac{2}{T}\int\limits_0^{T} f(t)\cos(k \omega_1 t) dt, \quad b_k =
	\frac{2}{T}\int\limits_0^{T} f(t)\sin(k \omega_1 t) dt}$$
	$a_0$, $c_0$, $A_0$ sind \textit{Konstanten}, $\omega_1$ ist die
	\textit{Grundkreisfrequenz}, $a_k$ und $b_k$ sind die \textit{reellen
	Koeffizienten}, $c_k$ ist der \textit{komplexe Koeffizient}, $A_k$ ist die
	\textit{Amplitude} und $\varphi_k$ ist die \textit{Phase}.\\
	\fbox{
	\begin{tabular}{p{9cm}p{9cm}}
		$a_k = c_k + \bar{c_k} = 2\Real(c_k) = A_k \cos(\varphi_k)$ &
		$b_k = j(c_k + \bar{c_k}) = -2\Imag(c_k) = -A_k \sin(\varphi_k)$ \\
		$c_k = \frac{a_k-jb_k}{2} = \frac{A_k}{2} e^{j\varphi_k} = \frac{\pi}{T}F(j k \omega)$ &
		$c_{-k} = \overline{c_k} = \frac{a_k+jb_k}{2} = \frac{A_k}{2} e^{-j\varphi_k}$ \\
		$A_k = 2|c_k| = \sqrt{a_k^2+b_k^2}$ & $\varphi_k =  \arg(c_k)$ oder unten $\downarrow$\\
	\end{tabular}}\\

	\textbf{Berechnung von $\varphi_k$ aus $a_k$ und $b_k$}\\
	\begin{tabular}{p{4cm}p{4cm}p{3cm}p{3.5cm}}
		$a_k> 0:$ & $\varphi_k = -\arctan(\frac{b_k}{a_k})$ &
		$a_k<0:$ &	$\varphi_k = -\arctan(\frac{b_k}{a_k}) + \pi$\\
		$a_k = 0 \wedge b_k > 0:$ &	$\varphi_k = -\frac{\pi}{2}$ &
		$a_k = 0 \wedge b_k < 0:$ &	$\varphi_k = \frac{\pi}{2}$\\
		$a_k = 0 \wedge b_k = 0:$ &	$\varphi_k = \text{nicht definiert}$
	\end{tabular}

	\subsection{Symmetrie}
		\begin{tabular}{|p{4.3cm}|p{4.3cm}|p{4.4cm}|p{4.4cm}|}
         	\hline
        	\textbf{gerade Funktion} & \textbf{ungerade Funktion} &
        	\textbf{Halbperiode 1} & \textbf{Halbperiode 2}\\
        	\hline
        	\includegraphics[width=3cm,trim=0 0 0 -5]{./bilder/gerade_funktion.png}&
        	\includegraphics[width=3cm]{./bilder/ungerade_funktion.png}&
 			\includegraphics[width=3cm]{./bilder/halbperiode_1.png}&   
			\includegraphics[width=3cm]{./bilder/halbperiode_2.png}\\
			\hline & & & \\			
   			$f(-t)=f(t)$ & $f(-t)=-f(t)$ & $f(t)=f(t+\pi)$ & $f(t)=-f(t+\pi)$\\
   			$b_k=0$ & $a_k=0$ & $a_{2k+1}=0$ & $a_{2k}=0$\\
   			$a_k = \frac{4}{T} \int\limits_0^{\frac{T}{2}} f(t) \cdot \cos(k \omega_1
   			t) dt$ &
   			$b_k =  \frac{4}{T} \int\limits_0^{\frac{T}{2}} f(t) \cdot
			\sin(k \omega_1 t) dt$ &
			$b_{2k+1}=0$ & $b_{2k}=0$\\
			\hline
      	\end{tabular}

	\subsection{Rechtecksignale}
	$$a_k=\frac{2}{T}\int\limits_{-t_1/2}^{t_1/2}A\cos\left(\frac{2\pi k}{T}t\right)dt=
	\left .\frac{2AT}{2\pi T k}\sin \left(\frac{2\pi k}{T}t\right)\right |_{-t_1/2}^{t_1/2}=
	\frac{2A}{\pi k}\sin\left(\frac{\pi t_1}{T}k\right)$$
	
	F"ur Verh"altnisse $\frac{T}{ggT(t_1,T)}=n\in\mathbb{N}$ verschwinden die
	$n.$ Harmonische und deren Vielfache.\\
	\includegraphics[width=19cm]{./bilder/fourierreihe-rechteck.png}
	
% Fourier-Integral / Fourier-Transformation
\section{Fourier-Transformation}
\begin{tabular}{|p{6cm} l|} \hline
	\textbf{Fouriertransformierte:} &
	$F(j\omega) = \int\limits_{-\infty}^{\infty} f(t)e^{-j\omega t}dt$ \\
	\textbf{Rücktransformierte:} &
	$f(t) = \frac{1}{2\pi}\int\limits_{-\infty}^{\infty}F(j\omega)e^{j\omega t}d\omega$ \\ \hline
\end{tabular} \\
\begin{tabular}{p{6cm} l}
Dies ergibt das \textbf{Korespondenzpaar:}	& $f(t) \laplace F(\omega)$ \\
											 & $F(t) \laplace 2\pi \cdot f(-\omega)$ \\
\end{tabular} \\
\begin{tabular}{p{6cm} l}
$F(\omega) = R(\omega) -jX(\omega)$ wobei &
$R(\omega) = \int\limits_{-\infty}^\infty f(t)\cdot \cos(\omega t)\,dt \quad \text{und} \quad X(\omega) =
\int\limits_{-\infty}^\infty f(t)\cdot \sin(\omega t)\,dt$ \\
 & $f(t)$ gerade: $X(\omega)$ verschwindet, f(t) ungerade: $R(\omega)$ verschwindet \\
\end{tabular} \\

Jede reelle $f(t)$ lässt sich aus Summe einer geraden und einer ungeraden Funktion beschreiben:\\
\begin{tabular}{lll}
$f(t) = f_e(t) + f_o(t)$ mit & $f_e(t) = \frac{1}{2}[f(t) + f(-t)]$ & $f_o(t) = \frac{1}{2}[f(t) - f(-t)]$ \\

Also: & $R(\omega) = 2 \int\limits_0^\infty f_e(t) \cos(\omega t)\,dt$ & $X(\omega) = 2 \int\limits_0^\infty
f_o(t) \sin(\omega t)\,dt$ \\

Und: & $f_e(t) = \frac{1}{\pi}\int\limits_0^\infty R(\omega)\cos(\omega t)\,d\omega$ & 
$f_o(t) = \frac{1}{\pi}\int\limits_0^\infty X(\omega)\sin(\omega t)\,d\omega$ \\
\end{tabular}

Bei \textbf{kausalen} Funktionen gilt:\\
$f_e(t) = f_o(t) = \frac{1}{2}f(t) \quad \quad \quad
f(t) = \frac{2}{\pi}\int\limits_0^\infty R(\omega) \cos(\omega t)\,dt = \frac{2}{\pi}\int\limits_0^\infty X(\omega)
\sin(\omega t)\,dt$

\begin{tabular}{|l|l|l|}
\hline
Spektraldichte / Spektraldarstellung	& $F(\omega)$ 		& KEINE absoluten Werte für Amplitude \& Phase \\
\hline
Amplitudendichte 						& $|F(\omega)| $		& f reell $\rightarrow$
$|F(\omega)|$ symetrisch zur Ordinatenachse
\\
\hline
Phasendichte							& $arg(F(\omega))$	& f reell $\rightarrow$ $arg(F(\omega))$ punktsymetrisch zum Ursprung \\
\hline
Kosinusamplitudendichte					& $R(\omega)$		& f reell $\rightarrow$ $R(\omega)$ gerade \\
\hline
Sinusamplitudendichte					& $X(\omega)$ 		& f reell $\rightarrow$ $X(\omega)$ ungerade \\
\hline
Dämpfung / Amplitudengang				& $A(\omega) = |H(\omega)|$ & $= \sqrt{H(\omega)\cdot \overline{H(\omega)}}$  \\
\hline
Phasenverschiebung						& $\Phi(\omega) = arg(H(\omega))$ & $= \arctan(\frac{Im(H(\omega))}{Re(H(\omega))})$ \\
\hline
Systemantwort							& $H(\omega) = A(\omega) \cdot e^{\jmath \Phi(\omega)}$ \\
\hline
\end{tabular}

\subsection{Symmetrie}
	Es gelten die gleichen Symmetrien wie bei der Fourierreihe.

\subsection{Hilbert-Transformation}
Das folgende Paar heisst \textit{Hilbert-Transformationspaar} und ermöglicht die Berechnung von Real- und Imaginärteil
einer Fouriertransformation auseinander. \\ 

\begin{tabular}{|l|} \hline
$\Re(\omega) = \frac{1}{\pi} \cdot \int\limits_{-\infty}^{\infty} \frac{\Im(u)}{\omega-u}du$ \\
$\Im(\omega) = -\frac{1}{\pi} \cdot \int\limits_{-\infty}^{\infty} \frac{\Re(u)}{\omega-u}du$ \\ \hline
\end{tabular}
\hspace{5mm}bzw. für analytische Signale: \hspace{5mm}
\begin{tabular}{|l|} \hline
$s_R(t) = - \frac{1}{\pi} \int\limits_{-\infty}^{\infty} \frac{s_I(u)}{t-u} du$ \\
$s_I(t) = \frac{1}{\pi} \int\limits_{-\infty}^{\infty} \frac{s_R(u)}{t-u} du$ \\ \hline
\end{tabular} \\

Allgemein ist die Hilbert-Transformation $\mathcal{H}$ folgendermassen definiert: \\
$\mathcal{H}(f(t)) := \frac{1}{\pi} \cdot \int\limits_{-\infty}^{\infty} \frac{f(u)}{t-u}du$ \\
Die Hilbert-Transformation eines Signals $s(t)$ ist gegeben durch die Faltung: \\
$\mathcal{H}(s(t)) = s(t) * \frac{1}{\pi t}$ mit $\frac{1}{\pi t} \: \laplace \: -j \cdot sgn(\omega)$ \\

\subsection{Eigenschaften}
		\begin{tabular}{|p{8cm}|p{8cm}|}
        	\hline
        	Linearität & 
        	$\alpha\cdot f(t) + \beta\cdot g(t) \laplace \alpha\cdot F(j\omega) +
        	\beta\cdot G(j\omega)$\\
        	\hline
			Zeitumkehrung (Spiegelung an der Y-Achse)&
			$f(-t) \laplace F(-j\omega) = F^*(jw)$ \\
			\hline        	
  			"Ahnlichkeit / Zeitskalierung &
  			$f(\alpha t) \laplace \frac{1}{|\alpha|}F \left (j\frac{\omega}{\alpha} \right)
  			\quad\alpha \in\mathbb{R}\setminus \{0\}$\\
  			\hline
  			Verschiebung im	Zeitbereich &
  			$f(t\pm t_0) \laplace F(j\omega)e^{\pm j\omega t_0}$\\
  			\hline
			Verschiebung im Frequenzbereich &
			$f(t)e^{\pm j\omega_0 t} \laplace F(j(\omega\mp\omega_0))$\\
			\hline
			Ableitung im Zeitbereich &
			$\frac{\partial^n f(t)}{\partial t^n} \laplace (j\omega)^n F(j\omega)$\\
			\hline
			Integration im Zeitbereich &
			$\int\limits_{-\infty}^{t}f(\tau)d\tau \laplace
			\frac{F(j\omega)}{j\omega}+F(0)\pi\delta(\omega)$\\
			\hline				
			Ableitung im Frequenzbereich &
			$t^n f(t) \laplace j^n \frac{\partial F(j\omega)}{\partial \omega^n}$\\
			\hline		
			Faltung im Zeitbereich &
			$f(t) \ast g(t) = \int\limits_{-\infty}^{\infty} f(\tau)g(t-\tau)d\tau \laplace
			F(j\omega) \cdot G(j\omega)$\\
			\hline
			Faltung im Frequenzbereich &
			$f(t) \cdot g(t) \laplace \frac{1}{2\pi}F(j\omega) \ast G(j\omega)$\\
			\hline
			Vertauschungssatz (Dualität) &
			$f(t) \laplace F(j\omega)\nonumber$ \\
 			& $F(t) \laplace 2\pi \cdot f(-j\omega)$\\
 			\hline
 			Modulation &
 			$\cos(\alpha t) \cdot f(t)  \laplace  \frac{1}{2}\cdot
 			\left[F(j(\omega-\alpha)) + F(j(\omega+\alpha))\right ]$\\
 			& $\sin(\alpha t) \cdot f(t) \laplace \frac{1}{2j}\cdot \left[
 			F(j(\omega-\alpha)) - F(j(\omega+\alpha))\right ]$\\
 			\hline
        	Parseval's Theorem &
 			$\int\limits_{-\infty}^{\infty}f(t)g^{\ast}(t)dt = \frac{1}{2\pi}
  			\int\limits_{-\infty}^{\infty}F(j\omega)G^{\ast}(j\omega)d\omega$\\
  			\hline
  			Bessel's Theorem &
  			$\int\limits_{-\infty}^{\infty}|f(t)|^2 dt = \frac{1}{2\pi}
  			\int\limits_{-\infty}^{\infty}|F(j\omega)|^2 d\omega$\\
  			\hline 			
			Anfangswerte &
			$f(0)=\frac{1}{2\pi}\int\limits_{-\infty}^{\infty}F(j\omega)d\omega
			\hspace*{1cm} F(0)=\int\limits_{-\infty}^{\infty}f(t)dt$\\
			\hline
			$\infty$ lange Folge von $\delta$-Impulsen &
			$\sum_{n=-\infty}^{\infty} \delta(t-n\cdot t_0) \laplace
			\sum_{n=-\infty}^{\infty} \frac{2\pi}{t_0}\delta(\omega-n\cdot
			\frac{2\pi}{t_0})$\\
			\hline
        \end{tabular}
        
\subsection{Beispiele}
\begin{tabular}{l l}
Rechteckimpuls $r_T$ der Breite $2T$ & $r_T \laplace \frac{2 \cdot \sin(\omega T}{\omega}$ \\
Signum-Funktion & $\frac{1}{\pi \cdot t} \laplace -j \cdot sgn(\omega)$ \\
				& $sgn(t) \laplace \frac{2}{j\omega}$ \\
\end{tabular}

\newpage
% Abtasttheoreme, diskrete Frouriertransformation
%Abtasttheorem und diskrete Fouriertransformation

\section{Abtasttheorem, diskrete Fouriertransformation \skript{59}}
	\begin{minipage}{12cm}
		Abtasten einer Funktion mit idealem Abtaster und Abtastintervall $\Delta t$.\\
		$ \scalebox{1.2}{$f(t) \cdot \delta_{\Delta t}(t) = \sum\limits_{k=-\infty}^{\infty} \Delta t \cdot f(k \Delta t) \cdot \delta(t-k \Delta t) $}$
	\end{minipage}
	\begin{minipage}{6cm}
		\begin{tabular}{|l l l|}
			\hline
				Periodisieren &$\laplace$ & Abtasten\\
				Abtasten & $\laplace$ & Periodisieren\\
			\hline
		\end{tabular}
	\end{minipage}

\subsection{Abtasttheoreme \skript{62}}
	\textbf{Kardinalreihe:}  $\scalebox{1.2}{$S(\omega) = \sum\limits_{k=-\infty}^{\infty} S(k\frac{\pi}{T}) \cdot \frac{\sin(\omega T - k\pi)}{\omega T - k\pi}$
	mit $S(k\frac{\pi}{T}) = 2Tc_k$}$\\

	\textbf{Abtasttheorem f\"ur die Frequenz:}
	F\"ur ein Signal von endlicher Dauer $2T$ ist die Fouriertransformierte durch ihre Abtastwerte an den Stellen $k\frac{\pi}{T}$
	vollst\"andig bestimmt. \\
	
	\textbf{Abtasttheorem von Shannon:}
	Ist ein Signal bandbegrenzt mit Grenzfrequenz $\omega_g$, so l\"asst sich das Signal anhand der Abtastwerte zu den Zeitpunkten
	$k\frac{\pi}{\omega_g}$ vollst\"andig rekonstruieren.
	
	
\subsection{Diskrete Fouriertransformation \skript{65}}
\begin{minipage}{14cm}
	mit $N$ Abtastpunkten\\
	-	Die DFT liefert nur $\frac{N}{2}$ unabh\"angige Koeffizienten, da $\hat{c_k}$ und $\hat{c_{-k}}$ konjugiert komplex sind.\\
	- Die Abtastfrequenz muss mindestens doppelt so gross sein, wie der zu beobachtende Frequenzinhalt...
\end{minipage}
\hspace{2em}
\begin{minipage}{6cm}
	$\boxed{\hat{c_k} = \frac{1}{N} \cdot \sum\limits_{n=0}^{N-1} y_n \cdot e^{-jkn\frac{2\pi}{N}}}$
\end{minipage}
	
	
	
	
\subsection{Alias-Effekt \skript{67}}

	Ist die Abtastfrequenz zu klein, k\"onnen die Frequenzen nicht mehr sauber getrennt werden, der sogenannte Alias-Effekt tritt auf:
	die Anteile aller Frequenzen werden der jeweils betragsm\"assig kleinsten passenden Frequenz zugeordnet.\\
	%	
	Summeneigenschaft: \qquad $\boxed{\hat{c_k} = \sum\limits_{m=-\infty}^{\infty} c_{k+mN}}$\\
	%
	\begin{minipage}[t]{8.5cm}
        \textbf{Beispiel:}\\
		\includegraphics[width=8cm]{./bilder/Alias-effekt.png}
	\end{minipage}
	\qquad
	\begin{minipage}[t]{10cm}
		\textbf{technische Anpassungsmöglichkeiten}\\
		\begin{itemize}
			\item Abtastfrequenz erhöhen
			\item Tiefpass $\Rightarrow$ obere Frequenzen werden abgeschnitten
		\end{itemize}
	\end{minipage}
	


% LaPlace-Transformation
\section{Laplace-Transformation}
	$$\boxed{F(s)=\int\limits_0^\infty f(t)e^{-st}dt} \qquad s=\sigma+j\omega$$\\
	- Definitionsbereich nur für kausale Systeme $t\geq 0$\\
	- Integrierbar über das Intervall $(0,\infty)$\\
	- Wachstum kleiner als der von eienr Exponentialfunktion\\ 
	- Gegen"uber $j\omega$ bei der Fourier-Transformation ist bei der
	Laplace-Transformation $s$ verallgemeinert zu $s=\sigma + j\omega$. Das
	bedeutet, dass die Fourier-Transformierte $F(j\omega)$ durch die
	Laplace-Transformation $F(s)$ ausgedr\"uckt werden kann.  	
  
 	\subsection{Eigenschaften}
  		\renewcommand{\arraystretch}{2}
		\begin{tabular}{|l|l|}
        	\hline
        	Linearität & 
 			$\alpha\cdot f(t) + \beta\cdot g(t) \laplace \alpha\cdot F(s) + \beta\cdot
 			G(s)$ \\
 			\hline
 			"Ahnlichkeit / Streckung &
 			$f(\alpha t) \laplace \frac{1}{\alpha}F \left (\frac{s}{\alpha} \right ) \quad 0
 			<\alpha \in\mathbb{R}$ \\
 			\hline
 			Faltung im Zeitbereich &
 			$f(t) \ast g(t) = \int\limits_{0}^{\infty} f(\tau)g(t-\tau)d\tau \laplace F(s)
 			\cdot G(s)$\\
 			\hline
 			Faltung im Frequenzbereich &
 			$f(t) \cdot g(t) \laplace \frac{1}{2\pi j}\int\limits_{c-j\infty}^{c+j\infty}
 			F(\xi) G(s-\xi)d\xi$ \\
 			\hline
 			Ableitung im Zeitbereich &
 			$\frac{\partial f(t)}{\partial t} \laplace sF(s)
 			-f(0+)$ \\
 			\hline
 			Ableitungen im Zeitbereich &
 			$\frac{\partial^n f(t)}{\partial t^n} \laplace s^nF(s)
 			-s^{n-1}f(0+)-s^{n-2}\frac{\partial f(0+)}{\partial t}-\ldots
 			-s^0\frac{\partial^{n-1} f(0+)}{\partial t^{n-1}}$ \\
 			\hline
 			Multiplikation mit $t$ &
 			$t\cdot f(t)  \laplace \frac{-\partial F(s)}{\partial s}$ \\
 			\hline
 			Ableitung im Frequenzbereich &
 			$(-t)^n f(t) \laplace  \frac{\partial^n F(s)}{\partial s^n}$ \\
 			\hline
 			Verschiebung im Zeitbereich &
 			$f(t\pm t_0) \laplace F(s)e^{\pm t_0 s}$ \\
 			\hline
 			Verschiebung im Frequenzbereich (Dämpfungssatz) &
 			$f(t)e^{\mp\alpha t} \laplace F(s\pm\alpha)$ \\
 			\hline
 			Integration &
 			$\int\limits_0^t f(\tau)d\tau \laplace \frac{F(s)}{s}$ \\
 			\hline
 			Anfangswert &
 			$\lim_{t\rightarrow 0} f(t) = \lim_{s\rightarrow \infty} sF(s),\text{~wenn
 			}  \lim_{t\rightarrow 0} f(t)\text{~existiert}.$ \\
 			\hline
 			Endwert &
 			$\lim_{t\rightarrow \infty} f(t) = \lim_{s\rightarrow 0} sF(s),\text{~wenn
 			}  \lim_{t\rightarrow \infty} f(t)\text{~existiert}.$ \\
 			\hline
       	\end{tabular}
		\renewcommand{\arraystretch}{\arraystretchOriginal}
	
	\subsection{Laplacetabelle}
	\begin{center}
		\begin{tabular}{|lcc|}
		\hline
			$\sigma \left( t \right)$ & $\laplace$ & $\frac{1}{s}$ \\
			$\sigma \left( t \right) \cdot t$ & $\laplace$ & $\frac{1}{s^2}$\\
			$\sigma \left( t \right) \cdot t^2$ & $\laplace$ & $\frac{2}{s^3}$\\
			$\sigma \left( t \right) \cdot t^n$ & $\laplace$ & $\frac{n!}{s^{n+1}}$\\
			$\sigma \left( t \right) \cdot e^{\alpha t}$ & $\laplace$ &
			$\frac{1}{s-\alpha}$\\
			$\sigma \left( t \right) \cdot e^{\alpha t}$ & $\laplace$ &
			$\frac{1}{( s - \alpha )^2}$\\
			$\sigma \left( t \right)\cdot t^2 \cdot e^{\alpha t}$ &
			$\laplace$ & $\frac{2}{{( s - \alpha )}^3}$\\
			$\sigma \left( t \right)\cdot t^n \cdot e^{ \alpha t}$ &
			$\laplace$ & $\frac{n!}{(s-\alpha)^{n+1}}$\\
			$\sigma \left( t \right) \cdot \sin \left(\omega t \right)$ & $\laplace$ &
			$\frac{\omega}{s^2 + {\omega^2}}$\\
			$\sigma \left( t \right) \cdot \cos \left( \omega t \right)$ & $\laplace$ &
			$\frac{s}{ s^2 + \omega^2}$\\
			$\delta \left( t \right)$ & $\laplace$ & $1\left( s \right)$ \\
			$\delta \left( t - \alpha \right)$ & $\laplace$ & $e^{- \alpha s}$\\
		\hline
		\end{tabular}
	\end{center}
		
	\subsection{Rücktransformation}
		\subsubsection{Vorgehen}
			\begin{tabular}{p{6cm}p{6cm}}
				1. Kürzen oder vereinfachen &
				3. Rücktransformation mittels Tabelle \\
				2. Partialbruchzerlegung falls nötig &
				4. $h(t)\hspace{0.2cm}\underline{nicht} < 0$ \\
			\end{tabular}
	
	\subsection{Lösung linearer Differentialgleichungen}
				\includegraphics[width=14cm]{./bilder/diffgleichungen.png}
				

% Diverses
\section{Diverses}
\subsection{Partialbruchzerlegung}
	\[f(x)=\frac{x^2+20x+149}{x^3+4x^2-11x-30} \Rightarrow \; \begin{array}{l}\text{Nenner faktorisieren mit}\\
	\text{Hornerschema, Binom, etc.}\end{array} \Rightarrow
	x^{3}+4x^{2}-11x-30=(x+2)(x^{2}+2x-15)=(x+2)(x+5)(x-3)\] Ansatz:
	\[f(x)=\frac{x^2+20x+149}{x^3+4x^2-11x-30}=\frac{A}{x-3} + \frac{B}{x+2} + \frac{C}{x+5}=
	\frac{A(x+2)(x+5)+B(x-3)(x+5)+C(x-3)(x+2)}{(x-3)(x+2)(x+5)}\]
	Gleichungssystem aufstellen mit beliebigen $x_i$-Werten (am Besten Polstellen oder 0,1,-1 wählen):
	\[\begin{array}{l}x_1=3:\;-9+60+149=A\cdot5\cdot8\;\;\;\Rightarrow A=5\\
	x_2=-2:\;-4-40+149=B(-5)\cdot3\; \Rightarrow B=-7\\
	x_3=-5:\;-25-100+149=C(-8)(-3) \Rightarrow C=1 \end{array} \Rightarrow
	f(x)=\frac{5}{x-3}-\frac{7}{x+2}+\frac{1}{x+5}\] weitere Ansätze für andere
	Typen von Termen: \[f(x)=\frac{5x^2-37x+54}{x^3-6x^2+9x}=\frac{A}{x}+\frac{B}{x-3}+\frac{C}{(x-3)^2}=\frac{A(x-3)^2+Bx(x-3)+Cx}{x(x-3)^2}\]
	\[f(x)=\frac{1,5x}{x^3-6x^2+12x-8}=\frac{A}{x-2}+\frac{B}{(x-2)^2}+\frac{C}{(x-2)^3}=\frac{A(x-2)^2+B(x-2)+C}{(x-2)^3}\]
	\[f(x)=\frac{x^2-1}{x^3+2x^2-2x-12}=\frac{A}{x-2}+\frac{Bx+C}{x^2+4x+6}=\frac{A(x^2+4x+6)+(Bx+C)(x-2)}{(x-2)(x^2+4x+6)}\]
			
\subsection{Hornerschema}
	\begin{minipage}[t]{9cm}
		- Pfeile $\Rightarrow$ Multiplikation\\
		- Zahlen pro Spalte werden addiert\\
		\includegraphics[width=6cm]{./bilder/hornerschema_1.png}\\
		$x_1 \Rightarrow$ Nullstelle (muss erraten werden!!)\\
		oberste Zeile = zu zerlegendes Polynom			
	\end{minipage}
	\begin{minipage}[t]{9cm}
		\textbf{Beispiel:}\\
		$f(x) = x^3-67x-126$\\
		\includegraphics[width=6cm]{./bilder/hornerschema_2.png}\\
		$\Rightarrow f(x) = (x-x_1)(b_2x^2 + b_1x + b_0) = (x+2)(x^2-2x-63)$	
	\end{minipage}

\subsection{Schrittfunktion - unit step}
	\begin{minipage}{10cm}
		$u(t) = \sigma(t) =	\begin{cases}
		  		 0 & \text{für } t < 0 \\
		  		 \frac{1}{2} \text{(praxis)}  \text{ oder undef. (math.)} & \text{für } t = 0 \\
		  		 1 & \text{für } t > 0
		  	\end{cases}
		$
		$\sigma(t) \laplace \frac{1}{j\omega} + \pi\delta(\omega) = \Sigma(\omega)$
	\end{minipage}
	\begin{minipage}{8cm}
		\includegraphics[width=6cm]{./bilder/unitstep.png}
	\end{minipage}

\subsection{Impulsfunktion - dirac delta function}
	\begin{minipage}{10cm}
		$\delta (t)=\begin{cases} 0 & t\ne 0\\\infty & t=0\end{cases} \qquad
		\text{und} \qquad \int\limits_{-\infty}^\infty \delta(t) \, \mathrm dt = 1 $\\
		$$\frac{du(t)}{dt}=\delta(t) \qquad
		\int\limits_{-\infty}^{\infty}\delta(t-t_0)f(t)dt=f(t_0)$$
	\end{minipage}
	\begin{minipage}{8cm}
		\includegraphics[width=6cm]{./bilder/diracimpulse.png}
	\end{minipage}
%\newpage
%\section{Riisä Tricks und Merksätze}
\begin{itemize}
  \item $H(s)$ = UTF = Laplacetransformierte der Impulsantwort ($y_\delta(t)$)
  \item $H(j \omega)$ = Frequenzgang = UTF auf imaginärer Achse
  \item $y_\sigma(t) = \int\limits_0^t y_\delta(u)du$
  \item $\left| \frac{ja + b}{a^2 + b^2} \right| = \sqrt{\frac{1}{a^2 + b^2}}$
  \item Dirac-Funktion: $s(t)\delta(t-t_0) = s(t_0)\delta(t-t_0)$
  \item Antwort eines LTI-Systems auf eine harmonisches Schwingung mit Frequenz $\omega \Rightarrow$ harmonische
  Schwingung mit gleicher Frequenz aber anderer Amplitude und Phase ($\mathcal{L}\{e^{j \omega t}\} = H(\omega) \cdot
  e^{j \omega t}$)
  \item $H(\omega)$ = komplexwertige Funktion der Frequenz $\omega$, die für jede Frequenz $\omega$ die
  Änderung von Amplitude und Phase durch das System speichert = Frequenzgang = Antwort auf harmonische Schwingung
  beliebiger Frequenz
\end{itemize}


% Wichtige Formeln
%\input{idiotenseite/trigo/trigoInclude}
%\section{Idiotenseite}
\input{idiotenseite/trigo/trigoInclude}
\input{idiotenseite/diverses/subsections/diverses}
\input{idiotenseite/diverses/subsections/Reihenentwicklung}
\input{idiotenseite/diverses/subsections/unbestimmteIntegrale}

% Integraltabelle
%\section{Integralrechnung}
Partielle Integration: $\int u(x) v'(x) dx = u(x)v(x) - \int u'(x) v(x) dx$

\subsection{Einige wichtige Integrale}
  	\renewcommand{\arraystretch}{2}
	\begin{tabular}{|l|l|}
    	\hline
    	$\int \sin(x)dx=-\cos(x)$ & $\int \sin(a+bx)dx=-\frac1b \cos(a+bx)$\\
    	\hline
	  	$\int \sin^2(x)dx=-\frac14 \sin(2x)+\frac x2$ 
    	& $\int
    	e^{ax+c}\sin(bx+d)dx=\frac{e^{ax+c}}{a^2+b^2}(a\sin(bx+d)-b\cos(bx+d))$\\
    	\hline
    	$\int \cos(x)dx=\sin(x)$ & $\int \cos(a+bx)dx=\frac1b \sin(a+bx)$\\
    	\hline
	  	$\int \cos^2(x)dx=\frac14 \sin(2x)+\frac x2$ 
    	& $\int
    	e^{ax+c}\cos(bx+d)dx=\frac{e^{ax+c}}{a^2+b^2}(a\cos(bx+d)+b\sin(bx+d))$\\
    	\hline
    	$\int e^x dx=e^x$ & $\int e^{ax}dx=\frac1a e^{ax}$\\
    	\hline
    	$\int xe^{ax}dx=\frac{1}{a^2} e^{ax}(ax-1)$ & $\int x^2 e^{ax} dx =
    	e^{ax}\left( \frac{x^2}{a} - \frac{2x}{a^2} + \frac{2}{a^3}\right)$ \\
    	\hline
    	$\int x^n e^{ax} dx = \frac{1}{a} x^n e^{ax} - \frac{n}{a} \int x^{n-1}
    	e^{ax} dx$ & \\
    	\hline
    \end{tabular}

\section{Differentialrechnung}
\subsection{Einige wichtige Differentiale}
\begin{multicols}{2}
	\begin{tabular}{|l|l||l|l|}
    	\hline
    	\textbf{Funktion} & \textbf{Ableitung} & \textbf{Funktion} &
    	\textbf{Ableitung}\\
    	\hline
    	\hline
    	$C \text{ (Konstante)}$ & $0$ & $x$ & $1$\\
    	\hline
    	$x^n$ & $nx^{n-1}$ & $\frac1x$ & $-\frac{1}{x^2}$\\
    	\hline
    	$\sqrt{x}$ & $\frac{1}{2\sqrt{x}}$ & $e^x$ & $e^x$\\
    	\hline
    	$e^{bx}$ & $be^{bx}$ & $a^x$ & $a^x \ln(a)$\\
    	\hline
    	$\ln(x)$ & $\frac1x$ & $\sin(x)$ & $\cos(x)$\\
    	\hline
    	$\cos(x)$ & $-\sin(x)$ & $\ln[f(x)]$ & $\frac{f^{'}(x)}{f(x)}$\\
    	\hline	
    \end{tabular}

\columnbreak  	
\subsection{Aufgabenbeispiel}
$f(t) =\sum_{k=0}^{\infty}\delta (t-k\pi) \cdot cos(t)$ \\
$ = \sum_{k=0}^{\infty} = \delta (t-k\pi) \cdot cos(\pi t)  \\
\; \laplace \; 
F(s) = \sum_{k=0}^{\infty} (-1)^k(e^{-k\pi s}) \\
= \sum_{k=0}^{\infty} (-1)^k(e^{-\pi s})^k$ 
\\ \\
Die Summenformel der geometrischen Reihe liefert: \\
$F(s) = \dfrac{e^{-\pi s}}{1 +e^{-\pi s}}$
\end{multicols}
\renewcommand{\arraystretch}{\arraystretchOriginal} \\
%\input{idiotenseite/diverses/subsections/unbestimmteIntegrale}

\end{document}