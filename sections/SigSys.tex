\section{Signale und Systeme}

	\begin{tabular}{|l|l|}
    	\hline
    	\textbf{Linearität} & \textbf{Zeitinvarianz}\\
    	\hline
    	$\mathcal{L}(x1+x2)=\mathcal{L}(x1)+\mathcal{L}(x2)$ & $\mathcal{L}(x(t)) = y(t)$ \\
    	$\mathcal{L}(c\cdot x)=c\cdot \mathcal{L}(x)$ & $\mathcal{L}(x(t-t_0)) = y(t-t_0)$ \\
		\hline    
    \end{tabular}
  	
	\subsection{Amplitudengang}
		$|H(\omega)| = \begin{cases}
			< 1 \text{ Dämpfung} \\
			> 1 \text{ Verstärkung}
		\end{cases}$
		
		
	\subsection{Lineare Systeme}
		\textbf{Basissignale}
		\begin{list}{$\bullet$}{\setlength{\itemsep}{0cm} \setlength{\parsep}{0cm} \setlength{\topsep}{0cm}} 
          \item Lineare Systeme sind durch die Antworten auf die
          Basissignale bestimmt.
          \item Basissignale müssen linear unabhängig voneinander sein, d.h. ein
		Basissignal darf nicht durch \textbf{Linearkombination} anderer Basissignale
		darstellbar sein          
		  \item Alle möglichen Eingangs-Funktionen müssen durch eine Linearkombination der
		Basissignale dargestellt werden können. $\Rightarrow$ \textbf{Periode des Eingangssignals =	Anzahl Basissignale}
        \end{list}
        \vspace{.2cm}
		\textbf{Berechnung der Systemantwort aufgrund der Basissignale und der
		Anregung}\\
		1. Eingangssignal $x$ als Linearkombination der Basisvektoren darstellen
		$\Rightarrow$ lineares Gleichungssystem\\
		$\Rightarrow x=r\cdot a + s\cdot b + t\cdot c\qquad$ ($x=$
		Eingangssignal; $a,b,c=$ Basisvektoren; $r,s,t=$
		Linearkombinationsparameter)\\ 
		2. Systemantwort $y=r\cdot \mathcal{L}(a) + s\cdot \mathcal{L}(b) + t\cdot \mathcal{L}(c); \qquad (\mathcal{L}(a)=$
		Systemantwort der Basis $a$)
	
	\subsection{Lineare zeitinvariante Systeme (LTI-Systeme)}
		LTI-Systeme sind durch ihre Impulsantwort $h$ vollständig bestimmt\\ \\	
		\textbf{Berechnung der Systemantwort von diskreten LTI-Systemen}\\
		$\; y=x*h \qquad$ ($y=$ Systemantwort; $x=$ Eingangssignal; $h=$
		Impulsantwort)\\
		
		\textbf{Berechnung der Systemantwort von kontinuierlichen LTI-Systemen}\\
		\begin{tabular}{ll}
			\parbox{8cm}{
			$$f_2(t) = h(t) * f_1(t) \; \laplace \; F_2(s) = H(s) \cdot F_1(s)$$
			$$h(t) \; \laplace \; H(s)$$}
			& \parbox{4cm}{
			\includegraphics[width=6cm]{./bilder/utf-theorie.png}}\\
		\end{tabular} \\

	\subsection{Eigenfrequenz, Frequenzgang}
	Zu jeder Frequenz gehört der eigene Frequenzgang mit $a, b \in \mathbb{C}$\\
	$$a e^{j\omega_1 t} + b e^{j\omega_2 t} \rightarrow a H(\omega_1) e^{j\omega_1 t} + b H(\omega_2) e^{j\omega_2 t}$$
	Frequenzgang bei Spektraldarstellung: $$\sum_{k=-\infty}^{\infty} c_k e^{jk\omega t} \rightarrow \boxed{H(\omega)}
	\rightarrow \sum_{k=-\infty}^{\infty} c_k H(\omega k) e^{jk\omega t}$$			
	
	\subsection{Faltung}
	$y(t) = f(t)\ast g(t) = g(t) \ast f(t) = (f \ast g)(t) :=
	\int\limits_{-\infty}^\infty f(u) \cdot g(t-u)\,du =
	\int\limits_{-\infty}^\infty f(t-u) \cdot g(u)\,du $ \\
	($g(t-u)$ entspricht einer Spiegelung an der Y-Achse und ist um t nach rechts geschoben)\\
	
	Hat $g\left(t\right)$ \textbf{keine negative} Argumente dann gilt :
	$\left(g \ast f \right)\left(t\right)=\int\limits_{-\infty}^t f(u) \cdot
	g(t-u)\,du$\\
	Hat $f\left(t\right)$ keine negative Argumente (Einschaltvorgang) dann gilt :
	$\left(g \ast f \right)\left(t\right)=\int\limits_{0}^t f(u) \cdot
	g(t-u)\,du$\\
	Bei einer Faltung mit einer $\delta\left(t\right)$Funktion
	gilt:$f\left(t\right) \ast \delta\left(t\right) = f\left(t\right)$
	
	\begin{multicols}{2}
		\begin{tabular}{|l | l |}
				\hline
					Faltungssatz: & $f(t)\cdot g(t) \; \laplace \; F(\omega) \ast G(\omega)$ \\
					& $f(t) \ast g(t) \; \laplace \; \frac{1}{2\pi}F(\omega) \cdot G(\omega)$ \\
				\hline
			\end{tabular}\\ \\
		\textbf{grafische Interpretation:}
		\begin{enumerate}
  			\item einfacheres Signal ($g(t)$) an der Y-Achse spiegeln
  			\item Verschiebung um t nach rechts
  			\item Multiplikation und Integration der beiden Signale
		\end{enumerate}
		\columnbreak
		\textbf{Bestimmung der Grenzen bei $\int\limits_{-\infty}^\infty f(u) \cdot g(t-u)\,du$:}
		\begin{enumerate}
		  \item Koordinatensystem: X-Achse: t, Y-Achse: u
		  \item $f(u)$ unterteilen $\rightarrow$ Streifen parallel zur X-Achse
		  \item $g(t-u)$ $\rightarrow$ Streiffen parallel zur $45^{\circ}$-Geraden
		  \item In u-Richtung sind die Integralgrenzen für ein \\ bestimmtes t abzulesen
		\end{enumerate}
	\end{multicols}
	(Beispiel 4.2, Skript S.33)
	

