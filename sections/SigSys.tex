\section{Signale und Systeme}
	\begin{tabular}{|l|l|}
    	\hline
    	\textbf{Linearität} & \textbf{Zeitinvarianz}\\
    	\hline
    	$S(x1+x2)=S(x1)+S(x2)$ & $S(x(t-t_0)=S(x)\cdot x(t-t_0)$ \\
    	$S(c\cdot x)=c\cdot S(x)$ & \\
		\hline    
    \end{tabular}
  	
	\subsection{Lineare Systeme}
		\textbf{Basissignale}
		\begin{list}{$\bullet$}{\setlength{\itemsep}{0cm} \setlength{\parsep}{0cm} \setlength{\topsep}{0cm}} 
          \item Lineare Systeme sind durch die Antworten auf die
          Basissignale bestimmt.
          \item Basissignale müssen linear unabhängig voneinander sein, d.h. ein
		Basissignal darf nicht durch \textbf{Linearkombination} anderer Basissignale
		darstellbar sein          
		  \item Alle möglichen Eingangs-Funktionen müssen durch eine Linearkombination der
		Basissignale dargestellt werden können. $\Rightarrow$ \textbf{Periode des Eingangssignals =	Anzahl Basissignale}
        \end{list}
        \vspace{.2cm}
		\textbf{Berechnung der Systemantwort aufgrund der Basissignale und der
		Anregung}\\
		1. Eingangssignal $x$ als Linearkombination der Basisvektoren darstellen
		$\Rightarrow$ lineares Gleichungssystem\\
		$\Rightarrow x=r\cdot a + s\cdot b + t\cdot c\qquad$ ($x=$
		Eingangssignal; $a,b,c=$ Basisvektoren; $r,s,t=$
		Linearkombinationsparameter)\\ 
		2. Systemantwort $y=r\cdot S(a) + s\cdot S(b) + t\cdot S(c); \qquad (S(a)=$
		Systemantwort der Basis $a$)
	
	\subsection{Lineare zeitinvariante Systeme (LTI-Systeme)}
		LTI-Systeme sind durch ihre Impulsantwort $h$ vollständig bestimmt\\ \\
		\textbf{Berechnung der Systemantwort von diskreten LTI-Systemen}\\
		$\; y=x*h \qquad$ ($y=$ Systemantwort; $x=$ Eingangssignal; $h=$
		Impulsantwort)\\
		
		\textbf{Berechnung der Systemantwort von kontinuierlichen LTI-Systemen}\\
		\begin{tabular}{ll}
			\parbox{8cm}{
			$$s_2(t) = h(t) * s_1(t) \laplace S_2(s) = H(s) S_1(s)$$
			$$h(t) \laplace H(s)$$}
			& \parbox{8cm}{
			\includegraphics[width=5cm]{./bilder/utf-theorie.png}}\\
		\end{tabular}	
		
	\subsection{Faltung}
	$y(t) = f(t)\ast h(t) = h(t) \ast f(t) = (f \ast h)(t) = \int\limits_{-\infty}^\infty x(u) \cdot h(t-u)\,du 
	= \int\limits_{-\infty}^\infty x(t-u) \cdot h(u)\,du $ \\
	wobei gilt: $h(t) = $ Impulsantwort des Systems \\
	\begin{multicols}{2}
		\textbf{grafische Interpretation:}
		\begin{enumerate}
  			\item einfacheres Signal an der Y-Achse spiegeln
  			\item Verschiebung um t nach rechts
  			\item Multiplikation und Integration der beiden Signale
		\end{enumerate}
		\columnbreak
		\textbf{Bestimmung der Grenzen:}
		\begin{enumerate}
		  \item Koordinatensystem: X-Achse: t, Y-Achse: u
		  \item Erster Faktor $\rightarrow$ Streifen parallel zur X-Achse
		  \item Zweiter Faktor $\rightarrow$ Streiffen parallel zur $45^{\circ}$-Geraden
		  \item Grenzen für ein bestimmtes t ablesen
		\end{enumerate}
	\end{multicols}	