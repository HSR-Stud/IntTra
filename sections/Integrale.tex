\section{Integralrechnung}
Partielle Integration: $\int u(x) v'(x) dx = u(x)v(x) - \int u'(x) v(x) dx$

\subsection{Einige wichtige Integrale}
  	\renewcommand{\arraystretch}{2}
	\begin{tabular}{|l|l|}
    	\hline
    	$\int \sin(x)dx=-\cos(x)$ & $\int \sin(a+bx)dx=-\frac1b \cos(a+bx)$\\
    	\hline
	  	$\int \sin^2(x)dx=-\frac14 \sin(2x)+\frac x2$ 
    	& $\int
    	e^{ax+c}\sin(bx+d)dx=\frac{e^{ax+c}}{a^2+b^2}(a\sin(bx+d)-b\cos(bx+d))$\\
    	\hline
    	$\int \cos(x)dx=\sin(x)$ & $\int \cos(a+bx)dx=\frac1b \sin(a+bx)$\\
    	\hline
	  	$\int \cos^2(x)dx=\frac14 \sin(2x)+\frac x2$ 
    	& $\int
    	e^{ax+c}\cos(bx+d)dx=\frac{e^{ax+c}}{a^2+b^2}(a\cos(bx+d)+b\sin(bx+d))$\\
    	\hline
    	$\int e^x dx=e^x$ & $\int e^{ax}dx=\frac1a e^{ax}$\\
    	\hline
    	$\int xe^{ax}dx=\frac{1}{a^2} e^{ax}(ax-1)$ & $\int x^2 e^{ax} dx =
    	e^{ax}\left( \frac{x^2}{a} - \frac{2x}{a^2} + \frac{2}{a^3}\right)$ \\
    	\hline
    	$\int x^n e^{ax} dx = \frac{1}{a} x^n e^{ax} - \frac{n}{a} \int x^{n-1}
    	e^{ax} dx$ & \\
    	\hline
    \end{tabular}

\section{Differentialrechnung}
\subsection{Einige wichtige Differentiale}
\begin{multicols}{2}
	\begin{tabular}{|l|l||l|l|}
    	\hline
    	\textbf{Funktion} & \textbf{Ableitung} & \textbf{Funktion} &
    	\textbf{Ableitung}\\
    	\hline
    	\hline
    	$C \text{ (Konstante)}$ & $0$ & $x$ & $1$\\
    	\hline
    	$x^n$ & $nx^{n-1}$ & $\frac1x$ & $-\frac{1}{x^2}$\\
    	\hline
    	$\sqrt{x}$ & $\frac{1}{2\sqrt{x}}$ & $e^x$ & $e^x$\\
    	\hline
    	$e^{bx}$ & $be^{bx}$ & $a^x$ & $a^x \ln(a)$\\
    	\hline
    	$\ln(x)$ & $\frac1x$ & $\sin(x)$ & $\cos(x)$\\
    	\hline
    	$\cos(x)$ & $-\sin(x)$ & $\ln[f(x)]$ & $\frac{f^{'}(x)}{f(x)}$\\
    	\hline	
    \end{tabular}

\columnbreak  	
\subsection{Aufgabenbeispiel}
$f(t) =\sum_{k=0}^{\infty}\delta (t-k\pi) \cdot cos(t)$ \\
$ = \sum_{k=0}^{\infty} = \delta (t-k\pi) \cdot cos(\pi t)  \\
\; \laplace \; 
F(s) = \sum_{k=0}^{\infty} (-1)^k(e^{-k\pi s}) \\
= \sum_{k=0}^{\infty} (-1)^k(e^{-\pi s})^k$ 
\\ \\
Die Summenformel der geometrischen Reihe liefert: \\
$F(s) = \dfrac{e^{-\pi s}}{1 +e^{-\pi s}}$
\end{multicols}
\renewcommand{\arraystretch}{\arraystretchOriginal} \\