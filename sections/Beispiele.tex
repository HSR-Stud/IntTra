\section{Beispiele}
	
	\subsection{Lineare Differentialgleichung}

		$y'' + 8y' + 25y = \sigma{t} \cdot \sin(2t)$ mit $y(0) = 2, y'(0) = -1$\\
		
		$\sin(2t) \; \laplace \; \dfrac{2}{s^2+4}$ \\ $Y(s)(s^2+8s+25) = 2s+15+\frac{2}{s^2+4}$
		$\Leftrightarrow Y(s) = \frac{2s+15}{s^2+8s+25}+\frac{2}{(s^2+4)(s^2+8s+25)}=
		\underbrace{\frac{2s+15}{s^2+8s+25}}_\text{Eigenschwinung durch Anfangszustand} +
		\underbrace{\frac{As + B}{s^2+4}}_\text{stationärer Zustand} +
		\underbrace{\frac{Cs + D}{s^2+8s+25}}_\text{Eigenschwinung durch Einschalten}$
		
	\subsection{Eigenschwingungen}
		Aus der Eigenschwingung können die Nullstellen des charakteristischen Polynom $p(s)$ 
		direkt abgelesen werden. \\[5mm]
		\textbf{Beispiel:} \\
		$y(t) = \frac{1}{2} e^{-t} \sin(3t) - \frac{2}{3} e^{-2t} \cos(2t) = 
		\underbrace{\frac{1}{2} e^{\textcolor{red}{-t}} \frac{1}{2j}(e^{\textcolor{red}{3j}t}
			-e^{\textcolor{red}{-3j}t})}_{NS = EW = \textcolor{red}{-1 \pm 3j}} - 
		\underbrace{\frac{2}{3} e^{\textcolor{red}{-2}t} \frac{1}{2}(e^{\textcolor{red}{2j}t}
			+e^{\textcolor{red}{-2j}t})}_{NS = EW = \textcolor{red}{-2 \pm 2j}}$ \\\\
		Damit ist das char. Polynom $p(s) = (s-NS_1)(s-NS_2)\ldots(s-NS_n)$ \\
		Bei mehreren gemessenen Eigenschwingungen werden die char. Polynome multipliziert. \\
		Der stationäre Zustand ist $\lim\limits_{t\rightarrow\infty}y(t) = \frac{1}{p(0)}$ *(Endwertsatz) \\