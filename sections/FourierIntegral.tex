\section{Fourier-Transformation}
\begin{tabular}{|p{6cm} l|} \hline
	\textbf{Fouriertransformierte:} &
	$F(j\omega) = \int\limits_{-\infty}^{\infty} f(t)e^{-j\omega t}dt$ \\
	\textbf{Rücktransformierte:} &
	$f(t) = \frac{1}{2\pi}\int\limits_{-\infty}^{\infty}F(j\omega)e^{j\omega t}d\omega$ \\ \hline
\end{tabular} \\
\begin{tabular}{p{6cm} l}
Dies ergibt das \textbf{Korespondenzpaar:}	& $f(t) \laplace F(\omega)$ \\
											 & $F(t) \laplace 2\pi \cdot f(-\omega)$ \\
\end{tabular} \\
\begin{tabular}{p{6cm} l}
$F(\omega) = R(\omega) -jX(\omega)$ wobei &
$R(\omega) = \int\limits_{-\infty}^\infty f(t)\cdot \cos(\omega t)\,dt \quad \text{und} \quad X(\omega) =
\int\limits_{-\infty}^\infty f(t)\cdot \sin(\omega t)\,dt$ \\
 & $f(t)$ gerade: $X(\omega)$ verschwindet, f(t) ungerade: $R(\omega)$ verschwindet \\
\end{tabular} \\

Jede reelle $f(t)$ lässt sich aus Summe einer geraden und einer ungeraden Funktion beschreiben:\\
\begin{tabular}{lll}
$f(t) = f_e(t) + f_o(t)$ mit & $f_e(t) = \frac{1}{2}[f(t) + f(-t)]$ & $f_o(t) = \frac{1}{2}[f(t) - f(-t)]$ \\

Also: & $R(\omega) = 2 \int\limits_0^\infty f_e(t) \cos(\omega t)\,dt$ & $X(\omega) = 2 \int\limits_0^\infty
f_o(t) \sin(\omega t)\,dt$ \\

Und: & $f_e(t) = \frac{1}{\pi}\int\limits_0^\infty R(\omega)\cos(\omega t)\,d\omega$ & 
$f_o(t) = \frac{1}{\pi}\int\limits_0^\infty X(\omega)\sin(\omega t)\,d\omega$ \\
\end{tabular}

Bei \textbf{kausalen} Funktionen gilt:\\
$f_e(t) = f_o(t) = \frac{1}{2}f(t) \quad \quad \quad
f(t) = \frac{2}{\pi}\int\limits_0^\infty R(\omega) \cos(\omega t)\,dt = \frac{2}{\pi}\int\limits_0^\infty X(\omega)
\sin(\omega t)\,dt$

\begin{tabular}{|l|l|l|}
\hline
Spektraldichte / Spektraldarstellung	& $F(\omega)$ 		& KEINE absoluten Werte für Amplitude \& Phase \\
\hline
Amplitudendichte 						& $|F(\omega)| $		& f reell $\rightarrow$
$|F(\omega)|$ symetrisch zur Ordinatenachse
\\
\hline
Phasendichte							& $arg(F(\omega))$	& f reell $\rightarrow$ $arg(F(\omega))$ punktsymetrisch zum Ursprung \\
\hline
Kosinusamplitudendichte					& $R(\omega)$		& f reell $\rightarrow$ $R(\omega)$ gerade \\
\hline
Sinusamplitudendichte					& $X(\omega)$ 		& f reell $\rightarrow$ $X(\omega)$ ungerade \\
\hline
Dämpfung / Amplitudengang				& $A(\omega) = |H(\omega)|$ & $= \sqrt{H(\omega)\cdot \overline{H(\omega)}}$  \\
\hline
Phasenverschiebung						& $\Phi(\omega) = arg(H(\omega))$ & $= \arctan(\frac{Im(H(\omega))}{Re(H(\omega))})$ \\
\hline
\end{tabular}

\subsection{Symmetrie}
	Es gelten die gleichen Symmetrien wie bei der Fourierreihe.

\subsection{Hilbert-Transformation}
Das folgende Paar heisst \textit{Hilbert-Transformationspaar} und ermöglicht die Berechnung von Real- und Imaginärteil
einer Fouriertransformation auseinander. \\ 

\begin{tabular}{|l|} \hline
$\Re(\omega) = \frac{1}{\pi} \cdot \int\limits_{-\infty}^{\infty} \frac{\Im(u)}{\omega-u}du$ \\
$\Im(\omega) = -\frac{1}{\pi} \cdot \int\limits_{-\infty}^{\infty} \frac{\Re(u)}{\omega-u}du$ \\ \hline
\end{tabular} \\

Allgemein ist die Hilbert-Transformation $\mathcal{H}$ folgendermassen definiert: \\
$\mathcal{H}(f(t)) := \frac{1}{\pi} \cdot \int\limits_{-\infty}^{\infty} \frac{f(u)}{t-u}du$ \\

\subsection{Eigenschaften}
		\begin{tabular}{|p{8cm}|p{8cm}|}
        	\hline
        	Linearität & 
        	$\alpha\cdot f(t) + \beta\cdot g(t) \laplace \alpha\cdot F(j\omega) +
        	\beta\cdot G(j\omega)$\\
        	\hline
			Zeitumkehrung (Spiegelung an der Y-Achse)&
			$f(-t) \laplace F(-j\omega) = F^*(jw)$ \\
			\hline        	
  			"Ahnlichkeit / Zeitskalierung &
  			$f(\alpha t) \laplace \frac{1}{|\alpha|}F \left (j\frac{\omega}{\alpha} \right)
  			\quad\alpha \in\mathbb{R}\setminus \{0\}$\\
  			\hline
  			Verschiebung im	Zeitbereich &
  			$f(t\pm t_0) \laplace F(j\omega)e^{\pm j\omega t_0}$\\
  			\hline
			Verschiebung im Frequenzbereich &
			$f(t)e^{\pm j\omega_0 t} \laplace F(j(\omega\mp\omega_0))$\\
			\hline
			Ableitung im Zeitbereich &
			$\frac{\partial^n f(t)}{\partial t^n} \laplace (j\omega)^n F(j\omega)$\\
			\hline
			Integration im Zeitbereich &
			$\int\limits_{-\infty}^{t}f(\tau)d\tau \laplace
			\frac{F(j\omega)}{j\omega}+F(0)\pi\delta(\omega)$\\
			\hline				
			Ableitung im Frequenzbereich &
			$t^n f(t) \laplace j^n \frac{\partial F(j\omega)}{\partial \omega^n}$\\
			\hline		
			Faltung im Zeitbereich &
			$f(t) \ast g(t) = \int\limits_{-\infty}^{\infty} f(\tau)g(t-\tau)d\tau \laplace
			F(j\omega) \cdot G(j\omega)$\\
			\hline
			Faltung im Frequenzbereich &
			$f(t) \cdot g(t) \laplace \frac{1}{2\pi}F(j\omega) \ast G(j\omega)$\\
			\hline
			Vertauschungssatz (Dualität) &
			$f(t) \laplace F(j\omega)\nonumber$ \\
 			& $F(t) \laplace 2\pi \cdot f(-j\omega)$\\
 			\hline
 			Modulation &
 			$\cos(\alpha t) \cdot f(t)  \laplace  \frac{1}{2}\cdot
 			\left[F(j(\omega-\alpha)) + F(j(\omega+\alpha))\right ]$\\
 			& $\sin(\alpha t) \cdot f(t) \laplace \frac{1}{2j}\cdot \left[
 			F(j(\omega-\alpha)) - F(j(\omega+\alpha))\right ]$\\
 			\hline
        	Parseval's Theorem &
 			$\int\limits_{-\infty}^{\infty}f(t)g^{\ast}(t)dt = \frac{1}{2\pi}
  			\int\limits_{-\infty}^{\infty}F(j\omega)G^{\ast}(j\omega)d\omega$\\
  			\hline
  			Bessel's Theorem &
  			$\int\limits_{-\infty}^{\infty}|f(t)|^2 dt = \frac{1}{2\pi}
  			\int\limits_{-\infty}^{\infty}|F(j\omega)|^2 d\omega$\\
  			\hline 			
			Anfangswerte &
			$f(0)=\frac{1}{2\pi}\int\limits_{-\infty}^{\infty}F(j\omega)d\omega
			\hspace*{1cm} F(0)=\int\limits_{-\infty}^{\infty}f(t)dt$\\
			\hline
			$\infty$ lange Folge von $\delta$-Impulsen &
			$\sum_{n=-\infty}^{\infty} \delta(t-n\cdot t_0) \laplace
			\sum_{n=-\infty}^{\infty} \frac{2\pi}{t_0}\delta(\omega-n\cdot
			\frac{2\pi}{t_0})$\\
			\hline
        \end{tabular}
        
\subsection{Beispiele}
\begin{tabular}{l l}
Rechteckimpuls $r_T$ der Breite $2T$ & $r_T \laplace \frac{2 \cdot \sin(\omega T}{\omega}$ \\
Signum-Funktion & $\frac{1}{\pi \cdot t} \laplace -j \cdot sgn(\omega)$ \\
				& $sgn(t) \laplace \frac{2}{j\omega}$ \\
\end{tabular}
