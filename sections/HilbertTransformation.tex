\subsection{Hilbertransformation \skript{53}}

		Das folgende Paar heisst \textit{Hilbert-Transformationspaar} und ermöglicht die Berechnung von Real- und Imaginärteil
		einer Fouriertransformation auseinander. \\ 
		
		\begin{tabular}{|l|} \hline
			$\Re(\omega) = \frac{1}{\pi} \cdot \int\limits_{-\infty}^{\infty} \frac{\Im(u)}{\omega-u}du$ \\
			$\Im(\omega) = -\frac{1}{\pi} \cdot \int\limits_{-\infty}^{\infty} \frac{\Re(u)}{\omega-u}du$ \\ \hline
		\end{tabular}
		\hspace{5mm}bzw. für analytische Signale: \hspace{5mm}
		\begin{tabular}{|l|} \hline
			$s_R(t) = - \frac{1}{\pi} \int\limits_{-\infty}^{\infty} \frac{s_I(u)}{t-u} du$ \\
			$s_I(t) = \frac{1}{\pi} \int\limits_{-\infty}^{\infty} \frac{s_R(u)}{t-u} du$ \\ \hline
		\end{tabular} \\
		
		Allgemein ist die Hilbert-Transformation $\mathcal{H}$ folgendermassen definiert: \\
		$\mathcal{H}(f(t)) := \frac{1}{\pi} \cdot \int\limits_{-\infty}^{\infty} \frac{f(u)}{t-u}du$ \\
		Die Hilbert-Transformation eines Signals $s(t)$ ist gegeben durch die Faltung: \\
		$\mathcal{H}(s(t)) = s(t) * \frac{1}{\pi t}$ mit $\frac{1}{\pi t} \: \; \laplace \; \: -j \cdot sgn(\omega)$ \\
		
			\begin{tabular}{| l | l l |}
				\hline
					Fourier-Transformierte: & $F(\omega) = \mathfrak{R}(\omega) + \im \mathfrak{I}(\omega)
					= \frac{1}{\pi \im} \int\limits_{-\infty}^{\infty}\frac{\mathfrak{R}(u) + \im \mathfrak{I}(u)}{\omega - u} du$ & \\
				\hline
					Hilbert-Transformationspaar: & $\mathfrak{R}(\omega) = \frac{1}{\pi} \int\limits_{-\infty}^{\infty} \frac{\mathfrak{I}(u)}{\omega - u} du$ & (Realteil)\\
					& $\mathfrak{I}(\omega) = -\frac{1}{\pi} \int\limits_{-\infty}^{\infty} \frac{\mathfrak{R}(u)}{\omega - u} du$ & (Imaginärteil)\\
				\hline
					Hilbert-Transformation $\mathcal{H}$: & $\mathcal{H}(f(t)) = \frac{1}{\pi} \int\limits_{-\infty}^{\infty} \frac{f(u)}{t-u} du
					= f(t) * \frac{1}{\pi t}$ & \\
				\hline
					Analytisches Signal: & $s_R(t) = - \frac{1}{\pi} \int\limits_{-\infty}^{\infty} \frac{s_I(u)}{t-u} du$ & (Realteil)\\
					& $s_I(t) = \frac{1}{\pi} \int\limits_{-\infty}^{\infty} \frac{s_R(u)}{t-u} du$ & (Imaginärteil)\\
				\hline
			\end{tabular}\\
		
			Die Hilbert-Transformation eines Signals $s(t)$ ist gegeben durch die Faltung $s(t) * \frac{1}{\pi t}$ mit $\frac{1}{\pi t} \; \laplace \; -j \cdot sgn(\omega)$.\\
			Durch die Hilbert-Transformation lässt es sich einfach die Fourier-Transformierte berechnen, wenn der Realteil oder der Imaginärteil gegeben
			ist.